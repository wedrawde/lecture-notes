\begin{center}

Lecture 2

\end{center}

Extensive parameters are the basic variables that characterise a system in thermal equilibrium. They are collectively denoted as $X_i$ where $i$ ranges over all such variables.

\vspace{\baselineskip}

Some examples of these are:

\begin{itemize}

\item Energy (internal energy): $U$, $E$
\item Total mass: $M$
\item Volume: $V$
\item Number of particles: $N$
\item Charges like electric charge: $q$

\end{itemize}

\subsection{First Law of Thermodynamics}

The first law of thermodynamics states that extensive parameters are conserved. That is $\frac{d X_i}{dt} = 0$ in thermal equilibrium.

If we have 2 systems $S_1$ and $S_2$ with extensive parameters $\{ X_i^1\}$ and $\{ X_i^2 \}$. If we combine these to form $S = S_1 \cup S_2$ then the extensive parameters of $S$ are $X_i = X_i^1 + X_i^2$ once $S_1$ and $S_2$ attain equilibrium.

Conventionally, the first law is a statement of energy conversation. That is that change in energy = heat transferred + work done by the system. Or $\delta U = \delta Q - \delta W$. We will use the form $dU = T dS - P dV$ which is the total change in energy under an infinitesimal process.

Extensive parameters $X_i$ scale with the system size or volume $V$. In particular, if we scale up the volume of the system by $\lambda$ i.e. $V \rightarrow \lambda V$ the extensive parameters $X_i \rightarrow \lambda X_i$.

\subsection{Second Law of Thermodynamics}

Kelvin and Clausius: There does not exist a physical process whose sole effect is to convert heat into work. In a more modern statement: the entropy of a system is non-decreasing in any physical process.

Clausius introduced the concept of entropy. If the heat transfer in a process is $\delta Q$ then the systems gains entropy. This gives Clausius relation: $$dS = \frac{\delta Q}{T}$$

$\delta Q$ is the change of $Q$ in a process or $\delta Q = Q_2 - Q_1$. $dU$ is an exact differential. If $U$ is a function of some variables $\{y_i\}$ then $dU = \sum_i \frac{\partial U}{\partial y_i} d y_i$.

\subsection{Entropy}

For any thermodynamic system we can associate an extensive quantity $S$, which is a function of the other extensive parameters satisfying the constraint: Entropy is maximised as a function of $X_i$.

$\exists S(X_i)$ and $S(X_i)$ is non-decreasing in a physical process $\implies \delta S \geq 0$.

Specifying $S(X_i)$ completes the characterisation of thermodynamic system. This defining relation $S(X_i)$ is called the entropic fundamental relation. $S(X_i)$ is a homogeneous function of degree 1 (this is a consequence of $S$ being extensive). Therefore if $X_i \rightarrow \lambda X_i$ then $S(\lambda X_i) = \lambda S(X_i)$.

Entropic fundamental relation specifies thermodynamic equilibrium.

\subsubsection*{Example}

Consider a system whose extensive parameters are $U,V,N$. $S(U,V,N)$ specifies the system.

\begin{itemize}

\item[a)] $S(U,V,N) = x \sqrt{U^2 V N^2}$ is not a possible entropic function as $S(\lambda U, \lambda V, \lambda N) = c \sqrt{\lambda^2 U^2 \lambda V \lambda^2 N^2} = c \lambda^{5/2} \sqrt{U^2 V N^2}$.
\item[b)] $S(U,V,N) = c (U^2 V^4 N)^{1/7}$ is a possible function.
\item[c)] $S(U,V,N) = \exp \left (\frac{UV}{N^2}\right)$ is not as $S(\lambda U, \lambda V, \lambda N) = \exp \left (\frac{UV}{N^2}\right) = S(U,V,N)$.

\end{itemize}

\subsection{Intensive Parameters}

The intensive parameters (entropic) $Z_i$ are conjugate to the extensive parameters $X_i$. That is \\ $Z_i = \frac{\partial S}{\partial X_i}$.

$Z_i$ are homogeneous functions of degree 0. 

For example:

\begin{itemize}

\item Temperature ($T$) is conjugate to $U$
\item Pressure ($P$) is conjugate to volume $V$.

\end{itemize}

\subsection{Third Law of Thermodynamics}

The entropy change of any process at fixed temperature (isothermal) goes to zero as $T$ goes to zero. Or $\lim_{T \rightarrow 0} S(T, X_i) = 0$.