\begin{center}

Lecture 1

\end{center}

\section*{Preliminaries}

\url{http://www.maths.dur.ac.uk/~dma0mr/StatMech}

\subsection*{Books}

\begin{itemize}

\item K. Huang: Statistical Mechanics

\item R. K. Pathria: Statistical Mechanics

\end{itemize}

\subsection*{Outline of Course}

\begin{itemize}

\item Thermodynamics

	\begin{itemize}
	\item Applications of Thermodynamics
	\end{itemize}

\item Probability \& Random Walk

\item Ensemble theory (classical statistical mechanics)

\item Quantum statistics

\item Applications

	\begin{itemize}
	\item Bose \& Fermi statistics
	\item Bose-Einstein condensation
	\item Neutron stars, white dwarves and gravitational collapse
	\end{itemize}


\end{itemize}

\section{Thermodynamics}

Thermodynamics is a phenomenological theory: based on observations. Thermodynamics describes the physical behavior of macroscopic systems with many degrees of freedom. For example, gas in a box. There will be many air molecules, typically about $6.023 \times 10^{23}$ molecules.

To describe the behavior of the gas microscopically we require 3 position coordinates $x_i (t)$ and 3 velocity $v_i (t)$ for $i = 1 \ldots n$ where $n$ is the number of particles.

Typically one only wants to know some key macroscopic features of the system. For example: pressure ($p$), temperature ($T$). These variables are what thermodynamics describes.

The goal is to decipher physical properties from a small number of variables.

\begin{center}
\begin{tabular}{|c|c|}
\hline
Pros & Cons \\
\hline
Few parameters to describe complex systems & Ad-hoc formalism \\
\hline
Small set of principle (postulates) $\implies$ powerful results & Lose information about the details of the system \\
\hline
No need of detailed model of the physical system & \\ 
\hline
\end{tabular}
\end{center}

\subsection{Ideal Gas}

$N$ molecules of gas in a volume $V$. Empirically the ideal gas law $PV = N k_b T$ is upheld to very good accuracy. $k_b$ is Boltzmann's constant with value $1.38 \times 10^{-16} \text{erg}/\text{deg}$. 

\subsection{Equations of state}

General concept in thermodynamics giving a relation between $P$, $V$, $T$ (\& more generally other charges) such that $f(P, V, T) = 0$. In the case of the ideal gas: $PV - N k_b T = 0$ is the equation of state.

\subsection{Thermodynamical Postulates}

\begin{enumerate}

\item Macroscopic systems tend to evolve towards configurations (states) whose properties are independent of initial conditions (\& prior history). Moreover, these states are characterized by a small number of parameters. We call these equilibrium configurations.

Equilibrium is a steady state, where things do not change macroscopically.

States refer to configurations in the phase space (roughly speaking it is the collection of positions and velocities of the individual particles). Typically this is a $6N$ dimensional space. 

\end{enumerate}