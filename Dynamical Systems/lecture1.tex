\begin{center}

Lecture 1

\end{center}

\section*{Introduction}

\subsection*{What is a dynamical system?}

A dynamical system is a set of functions which depend on ``time'', a single variable $t$. Time can be discrete $t=0,1,2,3,\ldots$ and therefore our dynamical variable will look like $x(t)$ and thus $x(0)=x_0$, $x(1)=x_1$ and so forth. To specify how $x_i$ ($i \in \mathbb{Z}$) we need some `law'.

\subsubsection*{Example}

Variables $x_i$, $y_i$. Some initial conditions $x_0 = 1$ and $y_0 = 1$.

\vspace{\baselineskip}

``Law of evolution'' $y_n = x_{n-1}$ and $x_n = y_{n-1} + x_{n-1}$.

This is a discrete time dynamical system or a `recurence relation'. This dynamical system produces the fibonacci numbers.

\subsubsection*{Logistic Difference Equation}

This could be considered as a population model.  $ x_{n+1} = a x_n - b x_n^2$ with $a,b > 0$ and constants.

This can be rescaled to give: $$x_{n+1} = r x_n (1-x_n)$$

\vspace{\baselineskip}

Time can also be continuous. Variables like $x(t)$ and $y(t)$. A dynamical system will have initial conditions like $x(0) = 1$ and $y(0) = 2$. The law of evolution are now differential equations. For example: $\frac{dy}{dt} = - x^2$ and $\frac{dx}{dt} = y + xy^59$. 