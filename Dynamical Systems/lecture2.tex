\begin{center}

Lecture 2

\end{center}

There are very few linear equations. Most are far harder to solve. ODEs can be divided up as such:

\begin{enumerate}

\item Linear equations (easy to deal with but very few are useful)
\item Non-linear equations (many of them but not easy to solve)
\item Integrable equations (very very few non linear ones which have a solution)

\end{enumerate}

\subsubsection*{Example: 3 body problem}

The three body problem cannot be solved in exact form. However, qualitative analysis is important (so we know the planets aren't suddenly going to throw earth out of the solar system)

\subsubsection*{Example: Volterra-hotka}

The Volterra-hotka equations are for predator prey population simulations. $x$ is a population of foxes and $y$ is a population of rabbit. Now

\begin{align*}
\dot{x} &= x(by-a) \\
\dot{y} &= y(x-dx)
\end{align*}

With $a,b,c,d>0$ constants.

If $x=y=0$ then $\dot{x} = \dot{y} = 0$ also if $x = c/d$ and $y = a/b$. Also if $x < c/d \implies \dot{y} > 0$ and similarly $x > c/d \implies \dot{y} < 0$

\subsubsection*{Example: Lorenz system}

\begin{align*}
\dot{x} &= 10 (y-x) \\
\dot{y} &= 28x - y - xy \\
\dot{z} &= xy - \frac{8}{3} z
\end{align*}

\section{Basics of Dynamical Systems}

\subsection{Standard form for ODEs as dynamical systems}

\subsubsection{Example 1}

$$\ddot{x} + \dot{x}^2 \sin x + \log x = 0$$

Put $\dot{x} = y$. Now $$\dot{y} + y^2 \sin x + \log x = 0$$ these two equations act as a dynamical system.

\subsubsection{Example 2}

$$\dddot{x} + \ddot{x} \dot{x}^2 e^x + x^5 = 0$$

Now let $\dot{x} = y$ and $z = \dot{y} = \ddot{x}$. The original equation is now $$\dot{z} + z y^2 e^x + x^5 = 0$$ This produces the following dynamical system:

\begin{align*}
\dot{x} &= y \\
\dot{y} &= z \\
\dot{z} &= -z y^2 e^x - x^5
\end{align*}

\vspace{\baselineskip}

A stand form for a dynamical system is as follows: variables $x_1 (t), \ldots, x_n(t)$ with equations $\dot{x}_i (t) = F_i (x_1 (t), \ldots, x_n (t), t)$ for $i = 1 \ldots n$.

If $F_i$ do not depend explicitly on time $t$ i.e. $\dot{x}_i = F_i (x_1, \ldots, x_n)$. These are called autonomous dynamical systems.

It is usual to take $F_i$ to be $C_1$ (i.e. continuously differentiable).

\subsection{Phase Space}

Phase space $M$ is the space of all allowed configurations of the systems so if $(x_1, \ldots, x_n)$ we allow $x_i \in \mathbb{R}$ then $M = \mathbb{R}^n$.

\subsubsection{Example 1}

Consider $\dot{x} = y$, $\dot{y} = -x$. Now consider $\frac{d}{dt} (x^2 + y^2) = 2 x \dot{x} + 2 y \dot{y}$. Using the equations we see that $2 x \dot{x} + 2 y \dot{y} = 0$ and the phase space must be $\mathbb{R}^2$ because any values on a circle are permitted which is all of $\mathbb{R}^2$.