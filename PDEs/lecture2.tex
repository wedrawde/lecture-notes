\begin{center}

Lecture 2

\end{center}

\subsection{Well posed problems and Simple PDEs}

An important question with PDEs is what additional data are needed to give a unique solution?

Hadamard called a problem (PDE + data) well posed if:

\begin{enumerate}

\item It has a solution
\item The solution is unique
\item The solution depends continuously on the initial data.

\end{enumerate}

Continuous dependence on the data means small changes in the initial (or boundary) data lead to only small changes in the solution.

\subsubsection*{Example}

For Laplace's equation, let $u(x,t)$ solve the problem

$$u_{tt} + u_{xx} = 0$$

With $t > 0$, $u(x, 0) = 0$ and $u_t (x, 0) = 0$.

Let $v(x, t)$ solve the problem:

$$v_{tt} + v_{xx} = 0$$

For $t > 0$ and $v(x, 0) = 0$ and $v_t (x, 0) = \epsilon \sin \frac{x}{\epsilon}$.

\vspace{\baselineskip}

The first problem has solution $u(x,t) = 0$. The second problem has solution $v(x,t) = \epsilon^2 \sin \frac{x}{\epsilon} \sinh \frac{t}{\epsilon}$.

Note that $u$ and $v$ both solve the same equation, and their initial time derivatives are close, in the send that $|| v_t (x, 0) - u_t (x, 0) ||_{\infty} = \epsilon$. 

But, at a later time, the maximum difference over $x$ is $|| v(x, t) - u(x, t) ||_{\infty} = \epsilon^2 | \sinh \frac{t}{\epsilon}$ which is exponentially large $\implies$ the problem is not well posed (or is ill posed).

\subsubsection*{Example}

Now supposed $u(x, t)$ and $v(x,t)$ solve the wave equation:

$$u_{tt} - u_{xx} = 0$$

With the same data as the previous example.

\vspace{\baselineskip}

Now the solutions are $u(x, t) = 0$, but $v(x, t) = \epsilon^2 \sin \frac{x}{\epsilon} \sin {t}{\epsilon}$.

Now $|| v(x, t) - u(x, t) ||_{\infty} = \epsilon^2 | \sin \frac{t}{\epsilon} | \leq \epsilon^2$. So now small changes in the initial data here lead only to small changes in the solution $\implies$ this is now a well posed problem.

This message of this: we need to be care to impose appropriate data if we want a well posed problem. Even a well posed problem may not have a ``global'' solution (i.e. throught the domain of interest).

\subsubsection*{Example}

Solve the ODE $y' = y^2$ with $y(0) = \alpha$ with $\alpha$ constant.

Separating variables gives: $$\int \frac{dy}{y^2} = \int dt \implies \frac{-1}{y} + \frac{1}{y(0)} = t - 0$$

Thus $$y = \frac{\alpha}{1 - \alpha t}$$

The solution blows up at the $ t = 1/\alpha$ and thus the solution only exists for a finite range of $t$.

\subsection{Simple PDEs}

\subsubsection*{Example}

Find the general solution $u(x, t)$ of $$u_{xx} = 0$$ and the particular solution satisfying $u(0, t) = 1$, $u(1, t) = \sin t$.

Integrate with respect to $x$ to get $$u_x = f(t)$$ and again to get $$u = x f(t) + g(t)$$ where $f, g$ are arbitrary functions.

The data give $u(0, t) = g(t) = 1$ and $u(1, t) = f(t) + 1 = \sin t$ and thus $f(t) = \sin t - 1$.

So the particular solution is $$u(x,t) = x (\sin t - 1) + 1$$