\begin{center}

Lecture 2

\end{center}

\subsection{Well posed problems and Simple PDEs}

An important question with PDEs is what additional data are needed to give a unique solution?

Hadamard called a problem (PDE + data) well posed if:

\begin{enumerate}

\item It has a solution
\item The solution is unique
\item The solution depends continuously on the initial data.

\end{enumerate}

Continuous dependence on the data means small changes in the initial (or boundary) data lead to only small changes in the solution.

\subsubsection*{Example}

For Laplace's equation, let $u(x,t)$ solve the problem

$$u_{tt} + u_{xx} = 0$$

With $t > 0$, $u(x, 0) = 0$ and $u_t (x, 0) = 0$.

Let $v(x, t)$ solve the problem:

$$v_{tt} + v_{xx} = 0$$

For $t > 0$ and $v(x, 0) = 0$ and $v_t (x, 0) = \epsilon \sin \frac{x}{\epsilon}$.

\vspace{\baselineskip}

The first problem has solution $u(x,t) = 0$. The second problem has solution $v(x,t) = \epsilon^2 \sin \frac{x}{\epsilon} \sinh \frac{t}{\epsilon}$.
 