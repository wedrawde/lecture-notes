\documentclass[a4paper,11pt]{article}
\usepackage{amssymb}
\usepackage{amsmath}
\usepackage{amsfonts}
\usepackage{fancybox}
\usepackage{graphicx}
\usepackage[lmargin=2cm, rmargin=2cm, bmargin=3cm, tmargin=2cm]{geometry}
\usepackage{multicol}
\usepackage{url}
\newcommand{\sech}{\mathrm{sech} \,}
\newcommand{\bra}[1]{\langle #1|}
\newcommand{\ket}[1]{|#1\rangle}
\newcommand{\braket}[2]{\langle #1|#2\rangle}
\newcommand{\set}[1]{\{#1\}}
\newcommand{\pd}[2]{\dfrac{\partial #1}{\partial #2}}
\newcommand{\od}[2]{\dfrac{d #1}{d #2}}
\newcommand{\Nab}{\vec{\nabla}}
\newcommand{\dop}[2]{\vec{#1}\cdot\vec{#2}}
\newcommand{\lap}{\nabla^2}
\newcommand{\dprime}{\prime\prime}
\title{Electromagnetism Lectures}
\author{Corey Grant}
\date{\today}
\begin{document}

\maketitle

Book: Griffiths, `Introduction to Electrodynamics', chapters 2 and 3 is this terms work, Electrostatics\\
Chapter 1 coveres the vector calculus.\\
\\
Introduction:\\
\\
Maxwell's Equations\\
\\
By the end of this course we aim to understand Maxwell's Equations.\\
Here they are:\\
$\vec{\nabla} \cdot \vec{E} = 4\pi \rho $\\
$\vec{\nabla} \cdot \vec{B} = 0$\\
$\vec{\nabla} \times \vec{E} = -\frac{1}{c} \frac{\partial \vec{B}}{\partial t}$\\
$\vec{\nabla} \times \vec{B} = \frac{1}{c} \frac{\partial \vec{E}}{\partial t} + \frac{4 \pi}{c} \vec{j}$\\
E is the electric field\\
B is the magnetic field\\
$\rho$ is the electric charge density\\
$\vec{j}$ is the current density\\
These represent a deep insight into electromagnetism, and have set the pattern for a lot of thinking in mathematical physics.\\
This course is on classical EM, but the quantum version of this theory describles all properties of matter down to $10^{-18}$m.\\
$\vec{E}$ and $\vec{B}$ are vector fields, defined throughout space. Maxwells equations relate the fields to the sources locally.\\
Units: Simon will work in gaussian units, Griffiths uses SI units.\\
Equations simplify in static case, no time dependance.\\
Equations for the electric and magnetic fields seperate and can be studied independantly\\
\\
Electric Charges and Currents\\
\\
Electric charge is fundamentally carried in quantized units by pointlike particles (electron, proton,...)\\
At a classical level, better to use a smooth electric charge density $\rho$, in statics only $\rho (\vec{x})$\\
The total charge in a region V is then $Q=\int_V \rho ~~dV$\\
An electric curretn is carried by moving charges. If the charge density $\rho$ has velocity field $\vec{v} (\vec{x}, t)$ then the current density is $\vec{j} = \rho \vec{v}$\\
$|\vec{j}|=\rho |\vec{v}|$\\
Physically, amount of charge passing through an element $\Delta S$ perpendicular $\vec{v}$ in time $\Delta t$\\
$\rho |\vec{v}| \Delta t \Delta S$\\
So $\vec{j}$ is charge per unit time\\
For an area element $\Delta S$ with normal $\vec{n}$\\
Current through element $\Delta S$ is $\vec{j} \cdot \vec{n} \Delta S$\\
So the total current through a surface S is\\
$I=\int_S \vec{j} \cdot \vec{n} ~~dS$\\
\\
Charge Conservation\\
\\
Total charge is conserved. Electric charge can be moved around, but it cannot be created or destroyed.\\
If total charge Q in a volume V changes, it must generate a current through the surface:\\
$-\frac{dQ}{dt} = I$\\
This is 
$-\frac{d}{dt} \int_V \rho ~~dV = \int_S \vec{j} \cdot \vec{n} ~~dS$\\
If V is not changing in time,\\
$\int -\frac{d \rho}{dt}~~ dV =\int_S \vec{j} \cdot \vec{n}~~ dS = \int_V \vec{\nabla} \cdot \vec{j}~~ dV$ (by div thm)\\
True for all volumes V, so:\\
$\vec{\nabla} \cdot \vec{j} = -\frac{d\rho}{dt}$\\
The continuity equation

\section{Electrostatics}

Coulomb's Law says:\\
$$F_{ij} = \frac{1}{4\pi \epsilon _0} \frac{q_i q_j}{r_{ij} ^2}~~~~~ (1.1.1)$$\\
$q_i$ are the charges\\
r is the distance\\
This was discovered experimentally in 1785. More precisely,\\
$$\vec{F}_{ij} = \frac{1}{4\pi \epsilon _0} \frac{q_i q_j}{r_{ij} ^3}\vec{r}_{ij}~~~~~ (1.1.2)$$\\
The force point in the direction of the displacement vector $\vec{r}_{ij}=\vec{r}_j - \vec{r}_i$\\
Units:$\frac{q_i q_j}{4\pi \epsilon}$ has units of $ML^3 T^2$\\
Lectures will be in gaussian units, where $\frac{1}{4\pi \epsilon _0} = 1$ meaning those units above define the units of charge.\\
Note that signs of $q_i,~q_j$ are the same, repulsive interaction.

\subsection{Principle of superposition}

The force due to a collection of charges is the sum of the forces dues to the individual charges considered individually:\\

$$\vec{F}_j = \sum_i \vec{F}_{ij} = \sum_i \frac{q_i q_j \vec{r}_{ij}}{r_{ij}^3} ~~~~~(1.1.3)$$

\subsection{Electric Field}

Coulomb's Law involves an 'action at a distance'. We want to replace this with a local interaction. To do this we introduce the electric field $\vec{E}(\vec{x})$.
Assert that the force on a charged q is $$\vec{F} = q\vec{E} ~~~~~(1.2.1)$$
To make this consistant with Coulomb's observation,
$$\vec{E}(\vec{r}) = \sum_i \frac{q_i (\vec{r} - \vec{r}_i)}{|\vec{r} - \vec{r}_i|^3} ~~~~~(1.2.2)$$
For statics, these two expressions for $\vec{F}$ are equivalent. Difference appears with a time dependance.\\
The electric field diverges at the location of the source charges $q_i$, which produces an infinite 'self-force' on the charges - note we excluded this by hand in (1.1.3). This can be seen classically as a problem of over-idealization, but returns in quantum theory.\\
For a classical treatment, shouldn't keep track of individual atomic charges. Sum over charges in small regions to construct a smooth charge density $\rho (\vec{x})$ s.t. total charge in small volume element $\Delta V$ is $\Delta q = \rho (\vec{r}^{~'})\Delta V$ where $\vec{r}^{~'}$ is the location of the element.\\
$\vec{E}(\vec{r}) = \sum _{\vec{r}^{~'}} \rho (\vec{r}^{~'}) \Delta V \frac{\vec{r} - \vec{r}^{~'}}{|\vec{r} - \vec{r}^{~'}|^3}$\\
$$\vec{E}(\vec{r}) = \int \rho (\vec{r}^{~'}) \frac{\vec{r} - \vec{r}^{~'}}{|\vec{r} - \vec{r}^{~'}|^3}~dV ~~~~~ (1.2.3)$$
Note: doesn't diverge for smooth $\rho (\vec{r}^{~'})$, and we can recover (1.2.2) for $\vec{r} \neq \vec{r}_i$ by taking $\rho (\vec{r}^{~'}) = 0$ for $|\vec{r}^{~'} - \vec{r}_i| > \epsilon$\\
\\
We can think of (1.2.2) as an approx to (1.2.3) for such sources, valid away from the sources.\\
Similarly, we are often interested in approxomations where $\rho (\vec{r}^{~'})$ is non zero on a line or a surface. We can then write:\\
$\vec{E}(\vec{r}) = \int_C \lambda (\vec{r}^{~'}) \frac{\vec{r} - (\vec{r}^{~'})}{|\vec{r} - (\vec{r}^{~'})|^3} ~dl$\\
and $\vec{E}(\vec{r}) = \int_S \sigma (\vec{r}^{~'}) \frac{\vec{r} - (\vec{r}^{~'})}{|\vec{r} - (\vec{r}^{~'})|^3} ~dS$\\
$\lambda (\vec{r}^{~'})$ is the linear charge density\\
$\sigma (\vec{r}^{~'})$ is the surface charge density\\
\\
Do questions 1-5 for own help\\
Problems Classes 27th Oct and 1st Dec 1pm E102

\subsection{Electric Flux and Gauss' Law}

The electric flux through a surface S is:\\
\\
Electric Flux $= \int _S \vec{E} . \vec{n} ~dS ~~~~~(1.3.1)$\\
\\
Consider a single point charge. The value of its charge at any point is $\vec{E} = \frac{q}{r^3}\vec{r}$\\
$\vec{E} \cdot \vec{n} ~dS = \frac{q}{r^2}\cos{\theta} ~ds$\\
$\cos{\theta} ~ds$ is the area of the surface element $\perp$ to $\vec{E}$\\
$=r^2 ~d\Omega $\\
$=r^2 d\theta \sin{\theta} d\phi $\\
So $\vec{E} \cdot \vec{n} ~dS = q~d\Omega = q \sin{\theta} ~d\theta ~d\phi ~~~~~(1.3.2)$\\
So $\int_S \vec{E} \cdot \vec{n} ~dS = \int _{unit~sphere} q d\Omega = \int ^{\pi} _0 ~d\theta ~\int^{2\pi} _0 ~d\phi ~ q \sin{\theta}=4\pi q ~~~~~(1.3.3)$\\
\\
If $d\Omega$ corresponds to more than 1 patch dS on S, it cancels, so still gives the same.\\
If q is outside, cancels to give 0.\\
\\
Principle of superposition helps here for multiple charges.\\
So, $\int_S \vec{E} \cdot \vec{n} ~dS = 4\pi \times$ charge enclosed by S.\\
$\int_S \vec{E} \cdot \vec{n} ~dS = 4\pi \int_V \rho ~dV ~~~~~ (1.3.4)$\\
This is Gauss' Law.\\
\\
Using the divergence theorem:\\
$\int \vec{\nabla} \cdot \vec{E} ~dV = 4\pi \int \rho ~dV$\\
True for any volume V, so $\nabla \cdot \vec{E} = 4\pi \rho ~~~~~(1.3.5)$\\
This is the first of Maxwell's Equations.\\
Take $\vec{\nabla} \cdot ( \int \rho (\vec{r}^{~'}) \frac{\vec{r} - \vec{r}^{~'}}{|\vec{r} - \vec{r}^{~'}|^3}~dV)$\\
$=  \int \rho (\vec{r}^{~'}) \nabla \cdot (\frac{\vec{r} - \vec{r}^{~'}}{|\vec{r} - \vec{r}^{~'}|^3}~dV)$\\
The evaluate this derivative $ \nabla \cdot (\frac{\vec{r} - \vec{r}^{~'}}{|\vec{r} - \vec{r}^{~'}|^3})$, take $\vec{r}^{~'}$ to be the origin.\\
So we want $\vec{\nabla} \cdot (\frac{\vec{r}}{|\vec{r}|^3})$. This = 0\\
But this implies $\vec{\nabla} \cdot \vec{E} = 0$ , not the answer we wanted.\\
The point is that $\frac{\vec{r}}{|\vec{r}|^3}$ blows up at r=0, so the calculation isn't valid there.\\
To understand whats going on, regulate divergence:\\
$\frac{\vec{r}}{|\vec{r}|^3} \rightarrow \frac{\vec{r}}{(|\vec{r}|+a)^3}$, $a\neq 0$\\
\\
To understand what happens at $r=0$ we regulate (above) and get its divergence to be $\dfrac{3a}{(r+a)^4}$\\
We note that:$\rightarrow 0$ as $a\rightarrow 0$ for $\vec{r}\neq \vec{0}$
$\rightarrow \infty$ as $a\rightarrow 0$ for $\vec{r} = \vec{0}$\\
Consider the integral:
$\int \vec{\nabla} \cdot \dfrac{\vec{r}}{(r+a^3)} ~dV = \int \dfrac{3a}{(r+a)^4} r^2 \sin{\theta} dr d\theta d\phi$\\
$4\pi \int _) ^\infty \dfrac{3ar^2 dr}{(r+a)^4}$,     $r=a(y-1)$\\
$4\pi \int_1 ^\infty \dfrac{3(y-1)^2 dy}{y^4} = 4\pi$\\
This is independant of the regulator used and in general it will be independant of the regulator as it is taken to 0.\\
Introduce a delta-function $\delta (x)$ defined by saying:\\
$\delta(x)=0$ for $x\neq 0$\\
$\int^a _{-a} \delta(x) ~dx =1$\\
$\lim _{a \rightarrow 0} \dfrac{3a}{(r+a)^4} = 4\pi \delta ^3 (\vec{r})$\\
Where $\int \delta^3(\vec{r})dV=1$ (in Cartesians $\delta^3(\vec{r})=\delta(x)\delta(y)\delta(z)$\\
Shift by a constant:\\
$\vec{\nabla}$ (wrt $\vec{r}$) $\cdot (\dfrac{(\vec{r}-\vec{r}^{'})}{|\vec{r} - \vec{r} ^{'}|^3})=4\pi \delta^3 (\vec{r} - \vec{r}^{'})~~~~~(1.3.8)$\\
Now $\vec{\nabla} \cdot \vec{E} = \int \rho(\vec{r}^{'}) \vec{\nabla} \cdot (\dfrac{(\vec{r}-\vec{r}^{'})}{|\vec{r} - \vec{r} ^{'}|^3})~dV^{'}$\\
$=4\pi \int \rho(\vec{r}^{'}) \delta ^3 (\vec{r} - \vec{r}^{'})~dV^{'}$\\
$=4\pi \rho(\vec{r})~~~~~(1.3.9)$\\
Note that $\vec{E}(\vec{r})$ depends on $\rho$ everywhere, while $\vec{\nabla} \cdot \vec{E}(r)$ depends only on $\rho(\vec{r})$\\
Application of Gauss' Law:\\
In symmetric situations, easy to determine $\vec{E}(\vec{r})$ using:
$$\int_S \vec{E}\cdot \vec{n} ~dS=4\pi Q_{encl}$$
Eg.1: Spherical Symmetry.\\
Suppose $\rho (\vec{r})=\rho(r)$, for some choice of origin.\\
1. Restrict $\vec{E}(\vec{r})$ using symmetry.\\
a.$\vec{E}$ will also depend only on r.\\
b.$\vec{E}=E(r)\vec{e}_r$: points radially, components tangental to the sphere are forbidden by the symmetry.\\
2. Apply Gauss' law to a suitible surface, i.e. a sphere:\\
$\int_S \vec{E}\cdot \vec{n} ~dS = 4\pi r^2 E$\\
$Q_{encl} = \int _V \rho (r) ~dV$\\
So $E(r) = \frac{Q_{encl}}{r^2} = \frac{1}{r^2} \int_{r^{'} < r} \rho(r^{'})~dV^{'} ~~~~~(1.3.10)$\\
e.g. consider a sphere of radius R, with uniform charge density $\rho$ carrying some total charge Q.\\
Then $E(r)= \frac{Q}{r^2}$ for $r>R$\\
Or $E(r) = \frac{1}{r^2} \int^r \rho(r^{'})~dV^{'}$ for $r<R$, meaning it $= \frac{\rho}{r^2} \int ^r dV^{'}$\\
$E(r)=\frac{\rho}{r^2}\frac{4\pi r^3}{3}=\frac{4\pi \rho r}{3}=\frac{Q}{R^3}r$ for some $r<R$\\
\\
Eg.2: Cylindrical Symmetry.\\
Suppose $\rho (\vec{r})=\rho(r)$
1. Restrict $\vec{E}(\vec{r})$ using symmetry.\\
a. $\vec{E}=\vec{E}(r)$
b. There is a discrete symmetry under $z \rightarrow -z$, $\theta \rightarrow -\theta$ (doing both preserves the coordinate system)\\
Assuming $\vec{E}$ is invariant under this, $\vec{E}=E(r)\vec{e}_r$\\
2. Apply Gauss' Law, with S being a cylinder.\\
Theres normals coming out of the top, bottom, and sides. The top and bottom ones come to zero, as the electric field and the normals are othogonal.\\
The other one gives us: $\int_S \vec{E}\cdot {n} ~dS=2\pi r l E$\\ 
as $\vec{E}\cdot\vec{n} = 0$ on the ends of the cylinder\\
so $E(r)= \frac{4\pi}{2\pi r l}Q_{encl} = \frac{4\pi}{2\pi rl} \int ^r \rho(r^{'}) dV^{'} = \frac{4\pi}{r}\int^r _0 \rho(r^{'}) dr^{'}$\\

\subsection{Electrostatic Potential}

$\vec{E}=q\dfrac{(\vec{r}-\vec{r}^{'})}{|\vec{r} - \vec{r}^{'}|^3}$\\
\\
Key fact, is that the curl of $\frac{\vec{E}}{q} = 0~~~~~(1.4.1)$\\
Proof is excercise 9. N.B. we show this even at $\vec{r}=\vec{r}^{'}$\\
So for a single charge the curl of the electric field is 0.\\
By principle of superposition, this will be true in general. More mathematically:\\
$\vec{E}=\int \rho(\vec{r}^{'}) \dfrac{(\vec{r}-\vec{r}^{'})}{|\vec{r} - \vec{r}^{'}|^3} ~dV^{'}$\\
$\vec{\nabla} \times \vec{E}(r)=\int \rho(\vec{r}^{'}) \vec{\nabla} \times\dfrac{(\vec{r}-\vec{r}^{'})}{|\vec{r} - \vec{r}^{'}|^3} ~dV^{'}=0~~~~~(1.4.2)$\\
This is another of the time independant Maxwell-eqns.\\
Integral form:\\
$\int_S (\vec{\nabla} \times \vec{E})\cdot \vec{n} ~dS =\oint_C \vec{E} \cdot d\vec{l}$ by Stokes Theorem.\\
So $\vec{\nabla}\times \vec{E} = 0 \Rightarrow \oint_C \vec{E} \cdot d\vec{l} = 0~~~~~(1.4.3)$\\
For a single charge, easy to check (1.4.3) (away from r=0)\\
\\
$\vec{E} = \frac{q}{r^2}\vec{e}_r~~~~~~~~d\vec{l}=dr \vec{e}_r + rd\theta \vec{e}_\theta +r\sin{\theta} d\phi \vec{e}_\phi=dx\vec{e}_x +dy\vec{e}_y +dz\vec{e}_z$\\
$\int^b _a \vec{E}\cdot d\vec{l}=\int^b _a \frac{q}{r^2} dr = \frac{q}{r_a}-\frac{q}{r_b}$\\
So $\oint \vec{E} \cdot d\vec{l} = 0$\\
Physical Interpretation: The force on a chrge q in $\vec{E}$ is $\vec{F}=q\vec{E}$.\\
so $q\int^b _a \vec{E} \cdot d\vec{l} =$ Work done against $\vec{E}$ moving q from a to b.\\
$\oint \vec{E} \cdot d\vec{l} = 0 \Rightarrow$ electrostatic force is conservative.\\
Equivalently, work done from a to b is independant of the path I follow: it only depends on a and b.\\
We can therefore define the electrostatic potential:\\
$\phi(\vec{r})=-\int ^{\vec{r}} _{\vec{a}} \vec{E} \cdot d\vec{l}~~~~~(1.4.4)$\\
Physically, $q\phi$ is the potential energy of a charge q at $\vec{r}$.\\
Note: its a minus sign because it is work done against the field. We notice that there is the usual ambiguity in the choice of the zero of $\phi$. Also, the path independance makes $\phi(\vec{r})$ a function.\\
$\phi(\vec{b})-\phi(\vec{a})-\int^{\vec{b}} _{\vec{a}} \vec{\nabla}\phi \cdot d\vec{l} = -\int^{\vec{b}} _{\vec{a}} \vec{E} \cdot d\vec{l}$\\
$$\vec{E}=-\vec{\nabla}\phi~~~~~(1.4.5)$$
This automatically satisfies $\vec{\nabla}\times \vec{E}=0$, as $\vec{\nabla}\times(\vec{\nabla} \phi)=0$ for any $\phi$\\
This also simplifes to work with a scalar.\\
For a single charge, $\vec{E} = \frac{q}{|\vec{r}|^3}\vec{r}$\\
Exercise: $\nabla(\frac{1}{|\vec{r}|}) = \frac{-\vec{r}}{|\vec{r}|^3}~~~~~(1.4.6)$\\
So for a single charge $\phi=\frac{q}{|\vec{r}|}~~~~~(1.4.7)$\\
Superposition:
$$\phi(\vec{r})=\int \dfrac{\rho(\vec{r}^{'})}{|\vec{r}-\vec{r}^{'}|} dV^{'}~~~~~(1.4.8)$$\\
Remark: In complicated spaces, itsw not always true that the curl of V=0 implies that its line integral is 0. If C has no surface S that spans it.
Hence, curl V does not imply path independance for some function f (we dont actually need to know this).\\
\\
Substituting $\vec{E}=-\vec{\nabla} \phi$ in $\vec{\nabla} \cdot \vec{E} =4\pi \rho$\\
We get Poisson's equation:
$$\nabla ^2 \phi = -4\pi\rho~~~~~(1.4.9)$$
\\
Field Lines and equipotential surfaces:\\
\\
The visualise the electric field, we can draw field lines: curves such that $\vec{E}$ is tangent to the curve at each point. The density of field lines carries information about the strength of the field.\\
Equipotential surfaces are surfaces of contstant electrostatic potential, i.e. $\phi$=constant.\\
Field lines are orthogonal to equipotentials, as $\vec{\nabla}\phi$ is normal to surfaces of constant $\phi$.\\
\\
Example: Calculate $\phi$ for a linear charge dist. of length $2d$ and linear charge density $\rho _{L}$.
We shall take the origin as the centre of the line, and use cylindrical polars.\\
$$\phi = \int \dfrac{\rho(\vec{r}^\prime)}{|\vec{r}-\vec{r}^\prime|} dV = \int^d _{-d} dz^\prime \dfrac{\rho_{L}}{|\vec{r}-\vec{r}^\prime|}=\int ^d _{-d} dz^\prime \dfrac{\rho_{L}}{[(r-r^\prime)^2(z-z^\prime)^2}]^\frac{1}{2} =\int^d _{-d} dz^\prime \dfrac{\rho _{L}}{[r^2 +(z-z^\prime)^2]^\frac{1}{2}}$$
We then substitute $t=z^\prime - z$\\
$$\int^{d-z} _{-d-z} \dfrac{\rho_{L}}{(r^2 + t^2)^\frac{1}{2}} ~dt$$
now letting $t=r\sinh\theta$, $dt=r\cosh\theta ~d\theta$\\
$$\int d\theta \rho_{L} = \rho_{L} arcsinh\frac{t}{r}$$
Now we get:\\
$\phi=\rho_{L} arcsinh\left(\dfrac{z^\prime -z}{r}\right) \left|^d _{-d} \right.$\\
$=\rho_{L} \ln\left(\dfrac{z^\prime - z}{r}+\sqrt{\frac{z^\prime-z)^2}{r^2}+1} \right )\left|^{z^\prime =d} _{z^\prime = -d} \right.$\\
$$=\rho_L \ln(\dfrac{\sqrt{(z-d)^2 +r^2}+d-z}{\sqrt{(z+d)^2 +r^2}-(d+z)})$$
For $z>>d,r$ then $(z-d)^2 >>r^2 \Rightarrow \sqrt{(z-d)^2 -r^2} \approx (z-d)(1+\frac{r^2}{2(z-d)^2})$\\
So $\phi \approx \rho_{L} \ln \frac{z+d}{z-d} \approx \rho_L \ln(1+\frac{2d}{z-d}) \approx \rho \frac{2d}{z-d}$\\ 
\\
Lecture on 3rd Nov canceled, Qs 11,15,25 for homework due 10th of Nov

\subsection{Dipoles}

Saw in previous lecture that far from sources, $\phi=\frac{Q_{total}}{r}$\\
But what if $Q_{total} = 0$?\\
i.e. in a hydrogen atom.\\
A dipole is the simplest example of a system with $Q_{total}=0$, two charges $-q,+q$ seperated by some small distance $\vec{\epsilon}$.\\
An ideal dipole is a limit as $\vec{\epsilon}\rightarrow 0$. To keep $\phi\neq0$ we will need to take $q\rightarrow \infty$, as we will see.\\
Dipole potential:\\
We will be working with spherical polars, with the line defined by the dipole having an angle $\theta$, origin directly between the two charges. This also makes it apparant that $\Phi$ (angle) is independant of $\phi$.\\
We have that:
$\phi=\frac{q}{r_{+}}-\frac{q}{r_{-}}~~~~~(1.5.1)$\\
$_{\pm} ^2 = r^2+(\frac{\epsilon}{2}^2 \mp 2r(\frac{\epsilon}{2}\cos\theta$\\
Consider small $\epsilon$:

$$\frac{1}{r_{\pm}}=\dfrac{1}{r(1+\frac{\epsilon^2}{4r^2} \mp\frac{\epsilon}{r}\cos\theta)^\frac{1}{2}}=\frac{1}{r}(1\pm\frac{\epsilon}{2r}\cos\theta+O(\frac{\epsilon^2}{r^2}))$$
So $\phi \approx \dfrac{q}{r}(1+\dfrac{\epsilon}{2r}\cos\theta)-\dfrac{q}{r}(1-\dfrac{\epsilon}{2r}\cos\theta)$\\
$=\dfrac{q\epsilon}{r^2}\cos\theta$ for $\epsilon <<r$\\
$=q\dfrac{\vec{\epsilon}\cdot \vec{r}}{|\vec{r}|^3}$\\
Take $\vec{\epsilon}\rightarrow 0$ holding $\vec{p}=q\vec{\epsilon}$ fixed:\\
The potential for an ideal dipole is
$$\phi=\dfrac{\vec{p}\cdot\vec{r}}{|\vec{r}|^3}~~~~~(1.5.2)$$
$\vec{p}$ is called the electric dipole moment.\\
Falls off more quickly, $\frac{1}{r^2}$ instead of $\frac{1}{r}$ for a point charge.\\
Electric Field: $\vec{E}=-\vec{\nabla}\phi,~~~~~\phi=\dfrac{\vec{p}\cdot\vec{r}}{|\vec{r}|^3}$\\
$\vec{\nabla}(\vec{p}\cdot\vec{r})=\vec{p}$\\
\\
$\vec{\nabla}\frac{1}{|\vec{r}|^2}=\frac{-3\vec{r}}{|\vec{r}|^5}$\\
\\
So $\vec{E}=3\frac{\vec{r}(\vec{p}\cdot \vec{r})}{|\vec{r}|^5}-\frac{\vec{p}}{|\vec{r}|^3}~~~~~(1.5.3)$\\
If we draw it, we get on the plane outwards from the origin, the field points down, whereas the axis along the dipole direction it points up. The electric field lines move up and outwards, then turn down. They pass the plain perpendicular to it.\\
Total electric flux through this would be zero, and the field always returns back to the same place.\\
Exercise: Derive equations for the field lines (or equipotentials) using $(1.5.2)$ or $(1.5.3)$

\subsection{Multipole expansions}

We want a simple characterization of the electrostatic potential $\phi$ of some localized charge density $\rho$ for $\vec{r}$ far from sources.\\
We saw that a point source has a $\phi$ that goes like $\frac{q}{r}$, and a dipole that goes like $\frac{1}{r^2}$\\
Similarly, we can introduce a quadropole by considering a charge distribution which has total charge of 0, and $\vec{p}=0$\\
By taking an appropriate limit, get point quadropole.\\
Exercise: check $\phi \approx \frac{1}{r^3}$\\
Consider a charge distribution $\rho(\vec{r}^\prime)$ with support in a volume V. We want to approximate $\phi(\vec{r})$ where $|\vec{r}|>>|\vec{r}^\prime|$ for all $\vec{r}^\prime$ in V.
$$\phi(\vec{r})=\int \dfrac{\rho(\vec{r}^\prime)}{|\vec{r}-\vec{r}^\prime|}~dV,~\text{as we know}$$
We want to expand $\dfrac{1}{|\vec{r}-\vec{r}^\prime|}$ in powers of $\dfrac{|\vec{r}^\prime|}{|\vec{r}|}$\\
Recall the Taylor series expansion:\\
$f(\vec{r}-\vec{r}^\prime)\approx f(x,y,z)-x^\prime\partial_xf-y^\prime \partial_y f- z^\prime\partial_z f+...$\\
$\approx f(\vec{r}) -\vec{r}^\prime \cdot \vec{\nabla}f(\vec{r})+\frac{1}{2}\vec{r}^\prime \cdot \vec{\nabla}(\vec{r}^\prime \cdot \vec{\nabla} f(\vec{r}))$\\
So $\dfrac{1}{|\vec{r}-\vec{r}^\prime|} \approx -\vec{r}^\prime \cdot \vec{\nabla}(\frac{1}{|\vec{r}|}) + \frac{1}{2} \vec{r}^\prime \cdot \vec{\nabla}(\vec{r}^\prime\cdot \vec{\nabla}\frac{1}{|\vec{r}|})~~~~~(1.6.1)$\\
$\vec{\nabla}(\frac{1}{|\vec{r}|})=\frac{-\vec{r}}{|\vec{r}|^3}$, so second approx is:
$\frac{1}{|\vec{r}-\vec{r}^\prime|}=\vec{1}{|\vec{r}|}+\frac{\vec{r}\cdot\vec{r}}{|\vec{r}^\prime|^3}+\frac{1}{2}\vec{r}^\prime\cdot\vec{\nabla}(\frac{-\vec{r}^\prime \cdot \vec{r}}{|\vec{r}|^3})$ is \\
$\frac{1}{2} x^\prime _i \pd{}{x_i}(\dfrac{-x^\prime _j x_j}{|\vec{r}|^3})=\frac{1}{2} x^\prime _i (\dfrac{3x_i x_j ^\prime x_j}{|\vec{r}|^5} - \dfrac{x^\prime _j \delta_{ij}}{|\vec{r}|^3})$\\
$=\frac{1}{2}x^\prime _i x^\prime_j (\dfrac{3x_ix_j}{|\vec{r}|^5}-\dfrac{\delta_{ij}}{|\vec{r}|^3})+\ldots$\\
$\dfrac{1}{|\vec{r}-\vec{r}^\prime|}=\dfrac{1}{|\vec{r}|}+\dfrac{\vec{r}\cdot\vec{r}^\prime}{|\vec{r}|^3}+\frac{1}{2}(\dfrac{3(\vec{r}^\prime\cdot \vec{r})^2}{|\vec{r}|^5}-\dfrac{\vec{r}^\prime\cdot \vec{r}^\prime}{|\vec{r}|^3})~~~~~(1.6.2)$\\
So:\\
$\phi(\vec{r})=\frac{1}{|\vec{r}|}\int \rho(\vec{r^\prime})~dV +\dfrac{\vec{r}}{|\vec{r}|^3} \cdot\int \rho(\vec{r^\prime})\vec{r}^\prime ~dV + \dfrac{1}{2|\vec{r}|^5}\int \rho(\vec{r}^\prime)(3x_i ^\prime x_j ^\prime x_ix_j-x_i^\prime x_i ^\prime x_jx_j) ~dV$\\
$=\dfrac{1}{\vec{r}}\int \rho(\vec{r}^\prime)~dV+\dfrac{\vec{r}}{|\vec{r}|^3} \cdot\int \rho(\vec{r}^\prime) \vec{r}^\prime ~dV +\dfrac{x_i x_j}{2|\vec{r}|^5}\int \rho(\vec{r}^\prime)(3x_i ^\prime x_j ^\prime -\delta_{ij}x_k ^\prime x_k ^\prime)~dV$\\
\\
$$\phi(\vec{r})=\dfrac{q}{|\vec{r}|}+\dfrac{\vec{r}\cdot\vec{p}}{|\vec{r}|^3} + \dfrac{q_{ij}x_i x_j}{2|\vec{r}|^5}+\ldots~~~~~(1.6.3)$$\\
Where $q=\int \rho(\vec{r}^\prime)~dV$ - total charge\\
$\vec{p}=\int \rho(\vec{r}^\prime)\vec{r}^\prime~dV$ - dipole moment\\
$q_{ij}=\int \rho(\vec{r}^\prime)(3x_i ^\prime x_j ^\prime - \delta_{ij}|\vec{r}^\prime|^2)~dV$ - Quadropole moment.\\
Note: Evaluating multiple moments of a series of integrals wrt $\vec{r}^\prime$ (over sources)\\
There is a big simplification, as the exact expression requires an independant integral for each $\vec{r}$\\
This is an expansion in $\dfrac{|\vec{r}^\prime|}{|\vec{r}|}$\\
We can see the monopole + dipole + ... structure we expected.\\
The quadropole moment is symmetric ($q_{ij}=q_{ji}$), and trace-free: $q_{11}+q_{22}+q_{33}=0$\\
\\
Example:\\
\\
Consider two point charges $-q,~q$ at $(0,0,0)$, $(0,0,d)$. (we will put the minus charged one at the origin)
Calculate q,$\vec{p}$, $q_{ij}$\\
$q=0$, as the total charge in the system is obviously 0.\\
$\vec{p}=q(0,0,d)$\\
$q_{11}=q(3\cdot0 - d^2)=-qd^2$\\
$q_{12}=q_{13}=0$ as $x^\prime = 0$ for all charges.\\
$q_{22}=-qd^2$\\
$q_{33}=q(3d^2-d^2)=2qd^2$\\
This fits our trace free condition as well.\\
If I put the charges at $(0,0,-\frac{d}{2}),~(0,0,\frac{d}{2})$ we get everything the same apart from the quadropole moment, which vanishes.\\
N.B. The multiple moments depend on your choice of origin.\\
(Exercise: work out how)

\subsection{Energy}

Consider first energy of charges in an external field (i.e. we will neglect the field that these charges generate).\\
Recall: $W=q\phi_{ext}(\vec{r})~~~~~(1.7.1)$\\
for a point charge $q$ at $\vec{r}$.\\
For a continuous distribution, sum over constituent charges. Charge in dV is $\rho(\vec{r})dV~\Rightarrow$ energy is $\rho(\vec{r})\phi_{ext}(\vec{r}~dV$\\
So total energy is
$$W=\int \rho(\vec{r})\phi_{ext}(\vec{r})~dV~~~~~(1.7.2)$$
If $\rho(\vec{r})$ is localized in a region of size R and $\phi_{ext}(\vec{r})$ is sufficiently slowly varying, then we can expand in Taylor Series:\\
$\phi_{ext}(\vec{r})=\phi(\vec{0})+\vec{r}\cdot \vec{\nabla}\phi(\vec{0})+\frac{1}{2}(\vec{r}\cdot \vec{\nabla})^2 \phi(\vec{0})+\ldots$\\
$=\phi(\vec{0})-\vec{r}\cdot \vec{E}(\vec{0})-\frac{1}{2}x_ix_j\pd{E_i}{x_j}(\vec{0})+\ldots$\\
Because this is an external field, $\vec{\nabla}\cdot\vec{E}=0$, that is,\\
$\delta_{ij}\pd{E_i}{x_j}=0$ (sources are outside the region I'm studying)\\
$(1.7.3)$ is equivalent to $\phi_{ext}(\vec{r})=\phi(\vec{0})-\vec{r}\cdot\vec{e}(\vec{0})-\frac{1}{2}(x_ix_j - \frac{1}{3}|r|^2\delta_{ij})\pd{E_i}{x_j}(\vec{0})~~~~~(1.7.4)$\\
Plugging $(1.7.4)$ into $(1.7.2)$:
$$W=\phi(\vec{0}\int\rho(\vec{r})~dV-\vec{E}\cdot \int\rho(\vec{r})\vec{r}~dV-\frac{1}{6}\pd{E_i}{x_j}\int(3x_ix_j-|r|^2\delta_{ij})\rho(\vec{r}~dV+\ldots$$
So
$$w=Q\phi(\vec{0})-\vec{p}\cdot\vec{E}-\frac{1}{6}q_{ij}\pd{E_i}{x_j}+\ldots~~~~~(1.7.5)$$
$\begin{array}{ccc}\text{Big Picture}&\phi&W\\\text{Charge} Q&\alpha\frac{Q}{r}&Q\phi_{ext}\\\text{dipole}\vec{p}&\alpha\frac{|\vec{p}|}{r^2}&\vec{p}\cdot{\vec{E}}\\\text{Quadropole} q_{ij}&\alpha \frac{|q|}{r^3}&\frac{-1}{2}q_{ij}\pd{E_i}{x_j}\\ \end{array}$\\
Aside: force on a diople 2 ways.\\
$W=-\vec{p}\cdot\vec{E}_{ext}(\vec{r})$\\
for a dipole at $\vec{r}$.\\
If I move the dipole $\delta\vec{r}$,\\
$\delta W=-\dop{F}{\delta r}$\\
$=-\vec{p}\cdot(\vec{E}(\vec{r}+\delta\vec{r})-E(\vec{r}))$\\
$=-\vec{p}\cdot(\dop{\delta r}{\nabla}(\vec{E}))$\\
$=-\delta\dop{r}{\nabla}(\dop{p}{E})$\\
So $\vec{F}=\Nab(\dop{p}{E})~~~~~(1.7.6)$\\
Alternatively, force on a point charge $q$ is $q\vec{E}$\\
An ideal dipole has a charge $-q$ at $\vec{r}$, $q$ at $\vec{r}+\vec{\epsilon}$, where $\vec{p}=q\vec{\epsilon}$ ($\vec{\epsilon}$ infintesimal)\\
$\vec{F}=q\vec{E}(\vec{r}+\vec{\epsilon})-q\vec{E}(\vec{r})$\\
$=q\dop{\epsilon}{\nabla}\vec{E}$\\
That is, $\vec{F}=\dop{p}{\nabla}\vec{E}~~~~~(1.7.7)$\\
In the component notation, (1.7.6) is:\\
$F_i=\pd{}{x_i}(p_j E_j)=p_j\pd{E_j}{x_i}$\\
(1.7.7) says:\\
$F_i=p_j\pd{}{x_j}E_j$\\
These would not be equal in general, but $\Nab \times \vec{E}=\vec{0}$\\
that is, $\pd{E_i}{x_j}-\pd{E_j}{x_i}=0$\\
This entered through the use of the elctrostatic potential $\phi$ in calculating $W$.\\
\\
Want to calculate is the energy of a charge distribution, taking into account $\vec{E}$ generated.\\
Bringing in charges from infinity, we find the first one uses no energy, but leaves an electric field, and we repeat.
$\vec{E}_1=\frac{q_1}{|\vec{r}|},~~~~W_1=0$\\
$W_2=\dfrac{q_1q_2}{|\vec{r}_2-\vec{r}_1|}$\\
$W_3=\dfrac{q_1q_3}{|\vec{r}_3-\vec{r}_1|}+\dfrac{q_2q_3}{|\vec{r}_3-\vec{r}_2|}$\\
carrying on, for a collection of point charges,
$$W=\sum^n _{i=1}\sum_{j<i}\dfrac{q_iq_j}{|\vec{r}_i-\vec{r}_j|}~~~~~(1.7.8)$$
$$=\dfrac{1}{2}\sum^n _{i,j=1~i\neq j} \dfrac{q_iq_j}{|\vec{r}_i-\vec{r}_j|}~~~~~(1.7.9)$$
For a continuous charge distribution:
$$W=\dfrac{1}{2}\int ~dV \int dV^\prime \dfrac{\rho(\vec{r})\rho(\vec{r}^\prime)}{|\vec{r}-\vec{r}^\prime|}~~~~~(1.7.10)$$
Note that $(1.7.10)$ can be motivated as a continuous limit of $(1.7.9)$, but I don't recover $(1.7.9)$ if I take $\rho(\vec{r})$ to be a sum of delta functions.\\
We will return to this later.\\
Now $\phi(\vec{r})=\int~dV^\prime \dfrac{\rho(\vec{r}^\prime)}{|\vec{r}-\vec{r}^\prime|}$, so $W=\frac{1}{2}\int~dV \rho(\vec{r})\phi(\vec{r})~~~~~(1.7.11)$\\
Note: Similarity to $W_{ext}=\int~dV \rho(\vec{r})\phi_{ext}(\vec{r})$ but factor of a ahlf to avoid double counting.\\
Now Maxwell's eqn says:
$$\rho=\dfrac{1}{4\pi}\dop{\nabla}{E}=-\dfrac{1}{4\pi}\nabla^2 \phi$$
$$W=\dfrac{1}{8\pi}\int~dV(\nabla^2\phi)\phi~~~~~(1.7.12)$$
Now integrate by parts(dropping the $\frac{1}{4\pi}$:\\
$$(\nabla^2\phi)\phi=(\Nab\cdot\Nab\phi)\phi=\Nab(\phi\Nab\phi)-\Nab\phi\cdot\Nab\phi$$
So
$$\int_{allspace}(\nabla^2\phi)\phi~dV=-\int_{allspace}\Nab\phi\cdot\Nab\phi~dV+\int_{allspace}\Nab\cdot(\phi\Nab\phi)~dV=-\int \dop{E}{E}~dV+\int_{boundary}\vec{n}\cdot(\phi\Nab\phi)dS$$
The final part vanishes, as $\phi$ does at large enough distance.\\
So 
$$W=\frac{1}{8\pi}\int~dV|\vec{E}|^2~~~~~(1.7.13)$$
Equivalent expression, but gives a different perspective: energy is in the electric field. $(1.7.13)$ generalizes nicely to the time dependant case.\\
Example: charged shell, of radius R, carrying a total charge Q over the surface, uniformly distributed.\\
$W=\frac{1}{2}\int~dV \int~dV^\prime \dfrac{\rho(\vec{r})\rho(\vec{r}^\prime)}{|\vec{r}-\vec{r}^\prime|}=\dfrac{1}{2}\int~dV\phi(\vec{r})\rho(\vec{r})$\\
For $r>R,~~\phi=\dfrac{Q}{r}$\\
$r<R,~~\phi=\dfrac{Q}{R}$\\
Charge density is $\rho=\dfrac{Q}{4\pi R^2}$, so $W=\dfrac{1}{2}\int R^2 \sin\theta ~d\theta~d\phi \cdot \dfrac{Q}{R}\cdot \dfrac{Q}{4\pi R^2}=\dfrac{Q^2}{2R}$\\
Alternatively, $\vec{E}(\vec{r})=\left\{\begin{array}{cc}\dfrac{Q}{r^2}\vec{e}_r&r>R\\0&r<R\end{array}\right.$\\
$W=\frac{1}{8\pi}\int~dV |\vec{E}|^2=\frac{1}{8\pi}\int^\infty _R ~dr r^2\sin\theta~d\theta~d\phi\dfrac{Q^2}{r^4}$\\
$W=\dfrac{Q^2}{2}\int ^\infty_R \dfrac{dr}{r^2}=\dfrac{Q^2}{2r}$\\
\\
Uniformly charged cylindrical charge distribution, radius R, infinitely lon, charge per unit length $\lambda$\\
We have to get the electric field from Gauss' law, by taking a bit of the cylinder.\\
$\vec{E}=E(r)\vec{e}_r$ by symmetry.\\
For $r>R$, $\int\vec{E}\cdot~d\vec{S}=4\pi Q_{encl}$\\
$E\cdot2\pi r L=4\pi\lambda L$\\
So $E=\dfrac{2\lambda}{r}$ for $r>R$\\
For $r<R$, $E2\pi rL=4\pi \dfrac{r^2}{R^2}\lambda L$\\
so $E=\dfrac{2r\lambda}{R^2}$\\
$W=\dfrac{1}{8\pi}\int |\vec{E}|^2~dV=\dfrac{1}{8\pi}\int^\infty _0 ~dr r~d\theta~dz ~E^2=\dfrac{L}{4}(\int^R_0r~dr\dfrac{4r^2}{R^4}+\int_R^\infty r~dr\dfrac{4\lambda^2}{r^2})=L\lambda^2[\frac{1}{4}\dfrac{r^4}{R^4}|^R _0 +\ln r|^\infty _R]=\infty$\\
\\
Self-Energy of a point charge\\
The point charge dormula $(1.7.9)$ doesnt include all the $(1.7.12)$, e.g. for 2 charges $-q,q$ $(1.7.9)$ gives:\\
$W=-\dfrac{q^2}{|\vec{r}_1-\vec{r}_2|}$\\
What we've taken out in $(1.7.9)$ is the self energy of a point charge.\\
For a single point charge $q$,
$\vec{E}=\dfrac{q}{r^2}\vec{e}_r$\\
$W=\dfrac{1}{8\pi}\int \dfrac{q^2}{r^4}r^2 \sin\theta~dr~d\theta~d\phi$\\
$W=\dfrac{1}{2}q^2\int^\infty _0 \dfrac{dr}{r^2}=-\dfrac{1}{2}q^2\left[\dfrac{1}{r}\right]^\infty _0=\infty$\\
Singularity, which is a short distance effect, associated with the point-like nature of the source.\\
Two attitudes:\\
-classically, pointlike charges are only an approximation.\\
-interested in relative energies of different configurations, not total energy.
$$W=W_{int}+W_{self-energy}$$
(for a collection of point charges)\\
$(1.7.9)$ is then $W_{int}$ (interaction)\\
Two charges, opposite charge:\\
$\vec{E}=\dfrac{q(\vec{r}-\vec{r}_1)}{|\vec{r}-\vec{r}_1|^3}-\dfrac{q(\vec{r}-\vec{r}_2)}{|\vec{r}-\vec{r}_2|^3}$\\
$|\vec{E}|^2=\dfrac{q^2}{|\vec{r}-\vec{r}_1|^4}+\dfrac{q^2}{|\vec{r}-\vec{r}_2|^4}-2\dfrac{q^2(\vec{r}-\vec{r}_1)\cdot(\vec{r}-\vec{r}_2)}{|\vec{r}-\vec{r}_1|^3|\vec{r}-\vec{r}_2|^3}$\\
$W_{int}=\dfrac{-q^2}{4\pi}\int\dfrac{(\vec{r}-\vec{r}_1)\cdot(\vec{r}-\vec{r}_2)}{|\vec{r}-\vec{r}_1|^3|\vec{r}-\vec{r}_2|^3}~dV$\\
Exercise: Evaluate this integral, which should be $\dfrac{4\pi}{|\vec{r}_1-\vec{r}_2|}$

\subsection{Conductors}

So far, we studied fixed charge distribution. In many interesting cases, $\rho$ is also affected by $\vec{E}$. Simple example, conductors-no other significant forces acting on the charges.\\
A conductor is a solid containing many free charge carriers. They can move anywhere within the solid, but cannot leave it. Electrons move in the opposite direction to the electric field. The polarization of the conductor creates an electric field, and this matches the external field, meaning there is no internal electric field.\\
If there exists a non zero $\vec{E}$ in a conductor, the force will move the electrons, which sets up a set charge distribution, which reduces $\vec{E}$.\\
Static configuration, therefore we have $\vec{E}|_{inside~conductor}=\vec{0}~~~~~(1.8.1)$\\
This will happen in around $10^{-19}$ seconds.\\
$\vec{E}=\Nab\phi$, so\\
$\phi|_{inside~conductor}=constant.~~~~~(1.8.2)$\\
Sufrace of a conductor is an equipotential surface, that is $\phi=const$ on the surface\\
Also, since $\dop{\nabla}{E}=4\pi\rho$,\\
$\rho|_{inside~conductor}=0~~~~~(1.8.3)$\\
That is, conductors only support surface charge densities. Can carry a net charge, but it will always be on the surface.\\
Finally, $\vec{E}=-\Nab\phi~~\Rightarrow~~\vec{E}|_{surface} \propto \vec{n}~~~~~(1.8.4)$\\
Electric field outside is normal to the surface of the conductor.\\
\\
Surface Charge Density $\sigma$:\\
If we consider some conductor, and locally there is a positive charge on the surface. The electric field all points away from the conductor, as there is no field inside. If we take a small gaussian surface that straddles the surface of the conductor with small thickness. The depth of this surface is negligable, and they dont get any electric flux anway. The area of the sides is not negligable. (actual explanation below)\\
Consider a small cylinder of area A and length $\delta l$ straddling the surface.\\
The ends are parallel to the surface of the conductor.\\
$\vec{E}=E\vec{n}$outside, 0 inside.\\
So $\oint_S \vec{E}\cdot\vec{n}~dS=EA=4\pi Q_{encl}$ (by gauss' law) $=4\pi\sigma A$. This means that 
$$E=4\pi \sigma~~~~~(1.8.6)$$
Note: No symmetry assumption here, use a small gaussian surface.\\
More generally, for a surface charge density not on a conductor, calling $\vec{E}^{+}$ as inside the surface, minus for outside:\\
$$\oint\dop{E}{n}~dS=(\vec{E}^{+}-\vec{E}^{-})\cdot \vec{n} A$$
Where $\vec{n}$ is pointing outwards.\\
This is still equal to $4\pi\sigma A$\\
So $(\vec{E}^{+}-\vec{E}^{-}=\Delta E_n=4\pi\sigma~~~~~(1.8.7)$\\
\\
The field in a cavity in a conductor:\\
Consider an empty cavity containing no charges. Whats $\vec{E}$ in the cavity?\\
Claim:$\vec{E}=\vec{0}$, $\phi=$constant inside the cavity$~~~~~(1.8.8)$\\
Suppose $\vec{E}\neq 0$ at some point inside the cavity$~~\Rightarrow~~\phi$ is not constant.\\
Since $\phi=$constant on the surface of the conductor, $\phi$ must have an extremum somewhere inside, say at the point p.\\
Over a small sphere surrounding p,\\
$\oint \vec{E}\cdot\vec{n}~dS=-\oint\Nab\phi\cdot\vec{n}~dS\neq 0$\\
But this is a contradiction, as we assumed that $\rho=0$ inside the cavity.\\
Hence, $\vec{E}=\vec{0}$, $\phi=$constant.\\
Note:\\
-This is uniqueness result - will see more later.\\
-This underlies electric shielding.\\
\\
Example 1:\\
Conductor sphere of radius a, total charge Q.\\
$\vec{E}=\dfrac{Q}{r^2}\vec{e}_r$, $r>a$
$\vec{E}=\vec{0}$ for $r<a$\\
$\phi=\dfrac{Q}{r}$ for $r>a$\\
The surface charge density $\sigma=\dfrac{Q}{4\pi a^2}$\\
Indeed, $\vec{E}(r-a)=\dfrac{Q}{a^2}=4\pi\sigma$\\
$\phi$ inside is equal to it at the surface, $\dfrac{Q}{a}$. If not our $\phi$ wouldnt be continuous, which it always is.\\
Voltage $V=\phi(a)=\dfrac{Q}{a}$\\
Capacitance $\dfrac{Q}{V}=a$\\
In general, capacitance depends only on the geometry of the conductor.\\
Note: for fixed $V$, $E=\frac{1}{a}$. This is also generally true if a conductor has some general shape, E biggest where radius of curvature is small.\\
\\
Example 2:\\
Same sphere as before, but this time with a concentric shell around it. Conductor for $r<R$, $a<r<b$.\\
Consider a sphere, radius R, charge Q, surrounded by a shell for $a<r<b$, charge 0.\\
Find $\vec{E}$, surface charge densities.\\
For $r<a$, it reduces to the previous problem. $\vec{E}=\vec{0}$ for $r<R$, $\dfrac{Q}{r^2}\vec{e}_r$ for $a>r>R$, $0$ for $b>r>a$, and $\dfrac{Q}{r^2}\vec{e}_r$ for $r>b$

\subsection{Potential Theory}

Develop general tools for solving Poisson's Eqn
$$\nabla^2 \phi=-4\pi\rho~~~~~(1.9.1)$$
This is useful for problems with conductors, but also more generally.\\
1) If $\rho=0$ (empty space), then $\nabla^2 \phi=0$, Laplace's eqn.\\
As we saw in the last lecture, solutions of $\nabla^2\phi=0$ have no maxima or minima (except on the boundry).\\
- Solutions `averaging' boundry values- interpolate boundry conditions with least possible variation.\\
2) I know a solution of $(1.9.1)$.
$$\phi(\vec{r})=\int \dfrac{\rho(\vec{r}^\prime)}{|\vec{r}-\vec{r}^\prime|}~dV^\prime~~~~~(1.9.2)$$
This is a solution for a specific boundry condition, namely $\phi\rightarrow0$ as $|\vec{r}|\rightarrow\infty$.\\
We might want to consider other boundry conditions, eg for conductors.\\
- Any two solutions $\phi_1,~\phi_2$ of $(1.9.1)$ must differ by a solution to Laplaces Eqn. If $\phi_1=\phi_2+\chi$\\
$$\nabla^2\chi=\nabla^2(\phi_2-\phi_1)=-4\pi\rho-(-4\pi\rho)=0$$
-$(1.9.2)$ satisfies $(1.9.1)$ because 
$$\nabla^2 \dfrac{1}{|\vec{r}-\vec{r}^\prime|}=-4\pi\rho\delta^{(3)}(\vec{r}-\vec{r}^\prime)~~~~~(1.9.3)$$
that is, $\dfrac{1}{|\vec{r}-\vec{r}^\prime|}$ is a Greens Function for $\lap$.\\
There are different Greens Functions for different boundry conditions.\\
\\
Uniqueness theorem:\\
Suppose $\lap \phi=-4\pi\rho$ in some volume V, and that $\phi$ is specified on the surface S bounding V: $\phi(\vec{r})=f(\vec{r})$ for $\vec{r}\in S$. This is called a Dirichlet boundry condition.\\
Then $\phi$ is unique.\\
Proof: by contradiction. Suppose $\exists \phi_1,~\phi_2$ which both solve $\lap \phi=-4\pi\rho$.\\
Set $u=\phi_1-\phi_2$\\
$\lap u =\lap\phi_1-\lap\phi_2=0$ in V, and $u(\vec{r})=\phi_1-\phi_2=f(\vec{r})-f(\vec{r})=0$ on S.\\
Now $\int_V \Nab\cdot(u\Nab u)~dV=\int u\Nab u \cdot \vec{n}~dS=0 $ as $u=0$ on S. (a)\\
But $\int\Nab \cdot (u\Nab u)~dV=\int u\lap u~dV+\int |\Nab u|^2~dV=\int |\Nab u|^2~dV$\\
So $\int|\Nab u|^2 ~dV=0$. But $|\Nab u|^2 \geq 0$.\\
$\Rightarrow |\Nab u|^2(\vec{r})=0$ at each point in V.\\
So $\Nab u=\vec{0}$, meaning $u$ must be a constant. As it is zero on the boundry, it must be zero everywhere.\\
This gives us that $\phi_1=\phi_2$.\\
Extensions:\\
1) If V is infinite, then a b.c. at infinity must be added, such that we still get 0 for (a)
$$\phi\approx\frac{1}{r}~\text{as}~r\rightarrow\infty~ \text{will do, as this}~\Rightarrow u\approx \frac{1}{r}$$
and
$$\int_{|r|=R} dS u\cdot \Nab u\approx R^2\frac{1}{R}\frac{1}{R^2}\rightarrow 0~\text{as}~R\rightarrow \infty$$
\\
2) We could instead fix $\vec{n}\cdot \Nab \phi$ on S (Neumann b.c.). This also makes in the integral (a) vanish.\\
\\
Note:\\
-$\phi$ will be unique up to a constant (as u=constant wont give $u=0$)\\
With Neumann b.c., there may be no solution - theorem assumed existance, it did not prove it.\\
\\
If we have a spherical conductor with voltage V, and $\phi\rightarrow\infty$ as $\vec{r}\rightarrow\infty$.\\
$\phi=V$ on a spherical conductor of radius a.\\
The moral of this is that the solution to the Dirichlet Problem is unique (if it exists). If you can find an answer, you know it is the right one.\\
$\phi=\frac{Va}{|\vec{r}|}$.\\
\\
Homework due Monday 5th December: Q23, 29, 32. Hint for 29 - use Q12 result.

\subsection{Method of Images}

Trick for solving problems with charges + conductors by finding a problem with just charges which has the same solution.\\
\\
Example: Conductor filling $x>0$, and a charge $q$ at $(-R,0,0)$. Conductor is earthed, so $\phi=0$ at $x=0$\\
Find $\phi$ for $x<0$. Find $\sigma$ on $x=0$.\\
That is, $\lap\phi=-4\pi\rho=-4\pi q\delta(x+r)\delta(y)\delta(z)$, with boundry conditions: $\phi=0$ at $x=0$, and $\phi\rightarrow 0$ at $r\rightarrow \infty$\\
In the absence of the conductor, $\phi=\frac{q}{|\vec{r}-\vec{r}^\prime|}$ where $\vec{r}^\prime=(-R,0,0)$\\
This solves Poissons equation, but not the boundry condition. We need to add something (image charge). Observe that
$$\phi=\frac{q}{|\vec{r}-\vec{r}^\prime}-\frac{q}{\vec{r}+\vec{r}^\prime}$$
Still solves Poissons Eqn for $x<0$, but not also satisfies the b.c.\\
Check at $x=0$, $|\vec{r}-\vec{r}^\prime|=\sqrt{R^2+y^2+z^2}=|\vec{r}+\vec{r}^\prime|$. So $\phi(x=0)=0$\\
This is the the potential for a charge q at $\vec{r}^\prime$, and an image charge $-q$ at $-\vec{r}^\prime$\\
Key Idea: $\exists$ a free space charge configuration where an equipotential has the geometry of the conductor in my given problem.\\
Find the charge distribution.\\
For $x<0$, $\vec{E}=-\Nab\phi$, for $x>0,~\vec{E}=0$\\
$4\pi\sigma=\dop{E}{n}$\\
Near $x=0$, $\vec{E}=E_x\vec{e}_x$\\
$E_x=-\pd{\phi}{x}=-\pd{}{x}(\dfrac{q}{\sqrt{(x+R)^2+y^2+z^2}}-\dfrac{q}{\sqrt{(x-R)^2+y^2+z^2}})$\\
$=\dfrac{q(x+R)}{((x-R)^2+y^2+z^2)^\frac{3}{2}}-\dfrac{q(x+R)}{((x+R)^2+y^2+z^2)^\frac{3}{2}}$\\
Set $x=0$: $E_x=\dfrac{2qR}{(R^2+y^2+z^2)^\frac{3}{2}}$\\
$\dop{E}{n}=-E_x$, as they are in the opposite direction.\\
$\sigma=\dfrac{-qR}{2\pi(R^2+y^2+z^2)^\frac{3}{2}}$\\
This forms a larger charge density close to the charge, and it gets smaller as the distance increases.\\
\\
Total charge: Do the integral in polar coordinates on the y-z plane.
$$Q=\int^{2\pi}_0~d\theta +\int^\infty_0 r~dr~\left[\dfrac{-qR}{2\pi(R^2+r^2)^\frac{3}{2}}\right]$$
Use $z=R^2+r^2$, $dz=2r~dr$
$$\Rightarrow -\frac{1}{2}\int^\infty_{R^2}\dfrac{qR}{z^\frac{3}{2}}~dz=-q$$
\\
See this by Gauss' Law.\\
$\oint \vec{E}\cdot d\vec{S}=\int_{hemisphere}\vec{E}~d\vec{S}$ as $E=0$ in the conductor. (a sphere centred on the projection of the point onto the conductor)\\
But $\psi$ has a dipole form, $\vec{E}\propto \frac{1}{r^3}$ as $r\rightarrow\infty$\\
So as we take the hemisphere larger, $\oint \vec{E}\cdot d\vec{S}\rightarrow 0$\\
Hence $Q_{encl}\rightarrow \infty$ as $r\rightarrow \infty$, so $Q_{plane}=-q$\\
\\
Energy: for 2 point charges, know $W=\frac{-q^2}{2d}$\\
So for image charge problem, $W=\frac{-q^2}{4R}$\\
Real problem has half the field, so has half the energy: $W=-\frac{q^2}{8R}$\\
\\
We can do problems with 2 planes meeting at an angle.\\
Firstly lets use a right angle as the meeting angle, and a charge $q$ outside of it, at position $(d_x,d_y)$:
Consider a charge $q$ at $(d_x,d_y)$ outside a conductor filling the region $x<0$ and $y<0$, held at $\phi=0$. Find $\phi$.\\
If we reflect like before on both conductors, we have to add another charge of the same charge at the same distance from the origin on the line from $q$ to the origin.\\
This has equipotentials at $x=0,y=0$ so it will give a $\psi$ in $x>0,y>0$ which satisfies the b.c.\\
N.B: Discrete symmetry under reflection and charge conjugation, which ensures that these are equipotentials.\\
If the angle is $\frac{\pi}{3}$, it has the same potential as the distibution with 5 image charges, arranged at the points of a hexagon.\\
If the angle is $\frac{2\pi}{3}$: Can it be done by method of images in general: Imposing reflection symmetry would generate image charges in the physical region. This means this one cannot be done.\\
If the distribution already has multiple charges outside the conductor, in the places we would need to put image charges for a single charge, it works.\\
\\
What if I took $V\neq0$ on the conductor? The surface of the conductor would not be held at zero, but at $v\neq0$.\\
This doesn't have a solution, as the conductor isn't grounded, and $\phi=0$ at $r$ tending to infinity is not satisfied.\\
\\
Now how about 2 conductors at $\phi=0$, parallel with a charge $q$ inbetween. We need an infinite amount of image charges, with alternating signs, so both sides look the same under reflection.\\
\\
For a conducting sphere, with a point charge q outside the sphere, $\phi=\dfrac{q}{|\vec{r}-\vec{R}|}-\dfrac{qb}{z|\vec{r}-\frac{b^2}{z^2}\vec{R}|}$
Where $|\vec{R}|=z$, $\vec{R}$ is the position of the charge, and b is the radius of the sphere. Proof is Q29.\\
What's the induced charge on the sphere?\\
Use Gauss' Law for equivalent charge distribution, with an image charge inside the conductor volume making a spherical equipotential (charge $\frac{-qb}{z}$. Taking gaussian surfaces outside of the sphere is the same for both with the conductor and without.\\
$4\pi Q_{encl}$ in both of them are the same, so $Q_{sphere}=-\dfrac{qb}{z}$.\\
\\
What is the force on the sphere due to the charge $q$?\\
We can equivalently ask what is the force on the charge due to the sphere (up to a sign).\\ 
This = force on $q$ due to the image charge (as $\vec{E}(\vec{r}=\vec{R})$ is the same)
$=\dfrac{qq^\prime}{|\vec{R}-\frac{b^2}{r^2}\vec{R}|^2}=\dfrac{-q^2b}{z(1-\frac{b^2}{z^2})^2z^2}=\dfrac{-q^2 bz}{(z^2-b^2)^2}$\\
\\
What about $V\neq0$ here? If we add $\phi=\dfrac{Vb}{\vec{r}}$, that is add a further image charge at $\vec{r}=\vec{0}$\\
\\
Force of a Conductor on itself:\\
Conductor with surface charge density $\sigma$, and Electric field $\vec{E}$, normal to surface of the conductor.\\
$\vec{E}=4\pi\sigma \vec{n}$\\
What force does this electric field exert on the conductor.\\
Might guess that the force per unit area is $\vec{E}\sigma$.\\
This is off by a factor of 2.\\
Get the factor right by considering the energy:\\
$W=\frac{1}{8\pi}\int|\vec{E}|^2~dV$. Focus on a small patch near conducting surface.\\
Consider a small area $\Delta A$, and the field for $r~\in~(0,r_0)$, where $0$ is the surface of the conductor.\\
$W=\frac{1}{8\pi}|\vec{E}|^2r_0~\Delta A$\\
$F=-\frac{dW}{dl}=\frac{dW}{dr_0}=\frac{1}{8\pi}|E|^2\Delta A=\frac{1}{2}|E|\sigma\Delta A$

\subsection{Two Dimensional Problems and Complex Potential}

For problems with a translational symmetry - independant of $z$, say, the probelm is effectively 2D, and for no sources, there is a relation to complex functions.\\
If we consider 2 parabolic plane conductors, reaching off with flat edges infinitely in z, facing each other, one with voltage 0, the other with voltage $V=V_0$\\
We can simply take this as a 2d-problem, as it is invariant in $z$. This gives us 2 parabolic conductors.\\
In 2d, Laplaces Eqn is
$$\pd{^2\phi}{x^2}+\pd{^2\phi}{y^2}=0$$
Want to solve this subject.\\
But source-free $\Rightarrow \dop{\nabla}{E}=0$\\
We can also wite that $\vec{E}=\Nab \times \vec{C}$\\
In 2-d, $\vec{C}=\psi \vec{e}_z$ without loss of generality, as $\vec{E}$ is independant in $z$\\
$E_x=\pd{\psi}{y}=-\pd{\phi}{x}$\\
$E_y=\pd{\psi}{x}=-\pd{\phi}{y}$\\
Cauchy-Riemann eqns.\\
So $f=\phi+i\psi$ is a complex analytic function of $x+iy$\\
Note - $\pd{^2\phi}{x^2}=\pd{^2\psi}{x\partial y}=-\pd{^2\phi}{y^2}$\\
For any complex analytic function $f$, $Re(f)$ (and $Im(f)$) are solutions to Laplace's eqns.\\
\\
Given a complex function $f$, we can construct a solution of Maxwell's eqns.\\
Ex 1: $f(z)=z~~\Rightarrow~~Re(f)=x$\\
$\vec{E}=-\vec{e}_x$\\
\\
Ex 2: $f(z)=z^2~~\Rightarrow~~\phi=x^2-y^2,~\psi=2xy$\\
$E_x=2x$\\
$E_y=2y$\\
Note that the field lines are the lines of constant $\psi$\\
This provides the solution for a situation with hyperbolic shaped conductors - this is called a quadropole lens.\\
\\
Eg 3: $f(z)=\sqrt{z}$\\

\subsection{Seperation of Variables}

Recall:\\
Laplace's equation in Cartesian co-ordinates is:
$$\nabla^2\phi=\pd{^2\phi}{x^2}+\pd{^2\phi}{y^2}{^2\phi}{z^2}=0~~~~~(1.12.1)$$
is a seperable equation.\\
That is, we can find some solution by assuming $\phi$ of product form:
$$\phi=X_{\sigma_x}Y_{\sigma_y}(y)Z_{\sigma_z}(z)~~~~~(1.12.2)$$
With\\
$X^{\prime\prime}+\sigma_xX=0$\\
$Y^{\prime\prime}+\sigma_yY=0$\\
$Z^{\prime\prime}+\sigma_zZ=0$\\
and $\sigma_x+\sigma_y+\sigma_z=0$ satisfies Laplaces Eqn $(1.12.1)$\\
\\
Example:\\
We might take:
$Y=e^{ik_y y},~~\sigma_y=k^2y$\\
$Z=e^{ik_z z},~~\sigma_z=k^2$\\
Then by definition $\sigma_x=-\sigma_y-\sigma_z=-(k^2y+k^2z)$\\
$\Rightarrow$~~$X=\alpha e^{\sqrt{k^2 y+k^2z}x}+\beta e^{-\sqrt{k^2y+k^2z}x}$
Usually products are not themselves useful, but we can try to find a solution by taking superpositions of products:
$$\varphi=\sum_{k_y,k_z}X_{k_yk_z}e^{ik_yy}e^{ik_zz}~~~~~(1.12.3)$$
This is useful, the the b.c. makes it hard even though any $\varphi$ can be broken down into this.\\
Where the boundry condition will determine:\\
- What values $k_y,k_z$ sum over\\
- $\alpha_{\vec{k}},\beta_{\vec{k}}$ of $X_{\vec{k}}$\\
We can also seperate variables in Laplace's Equation for cylindrical or spherical polars:\\
\\
Spherical Polars:
$$\nabla^2\phi=\frac{1}{r^2}\pd{}{r}(r^2\pd{\phi}{r})+\dfrac{r^2}{\sin\theta}\pd{}{\theta}(\sin\theta\pd{\phi}{\theta}+\dfrac{1}{r^2\sin^2\theta}\pd{^2\phi}{\varphi^2}~~~~~(1.12.4)$$
Assume that $R(r)\Theta(\theta)\Phi(\varphi)$\\
Plugging this in to give:
$$\frac{1}{R}\pd{}{r}(r^2\pd{R}{r})+\dfrac{1}{\Theta\sin\theta}\pd{}{\theta}(\sin\theta\pd{\Theta}{\theta})+\dfrac{1}{\sin^2\theta}\dfrac{1}{\Theta}\pd{^2\Phi}{\varphi^2}=0$$
Leading to the seperation of variables:
$$\od{^2\Phi}{\varphi^2}+m^2\Phi=0$$
$$\dfrac{1}{\sin\theta}\od{}{\theta}(\sin\theta\od{\Theta}{\theta})-\dfrac{m^2}{\sin^2\theta}\Theta+l(l+1)\Theta=0$$
$$\od{}{r}(r^2\od{R}{r})-l(l+1)R=0$$
If the domain of interest includes the whole sphere, $\varphi\in(0,2\pi)~~~\theta\in(0,\pi)$\\
Where $\Phi,\Theta$ are regular iff $m,l\in\mathbb{Z}~~|m|\leq l,l\geq 0$\\
\\
General Solution for spherical:
$$\varphi=\sum_{m,l}R_l(r)\Theta_{l,m}(\theta)\Phi_m(\varphi)$$
Where:\\
$R_l=A_lr^l+\dfrac{B_l}{r^{l+1}$\\
This is a useful technique if we can choose the individual product solution so that they satisfy some of the boundry conditions.\\
Example 1: Wave Guide:\\
If we have a cubeoid, with all but one side having a zero potential, then we can use cartesian seperation of variables.\\ 
For example 1:\\
We start by imposing some b.c. on individual product solns.\\
$\phi=0$ at $(1)y=0,a~~(2)z=0,b~~(3)x\rightarrow\infty$\\
$\phi=\phi_0$ at $x=0$ (also $(3)$)\\
Assume $\phi_\alpha=X_\alpha(x)Y_\alpha(y)Z_\alpha(z)$ satisfies $(1)$ and $(2)$ and $\lap\phi_\alpha=0$\\
I then have $Y^{\dprime}+k_y^2Y=0$\\
$Z^\dprime+k^2_zZ=0$\\
$X^\dprime-(k^2_y+k_z^2)X=0$\\
with our b.c.\\
So $Y(y)=\sin(\frac{n\pi y}{a})~~n\in1,2,3,\ldots~~k_y=\frac{n\pi}{a}$\\
$Z(z)=\sin(\frac{m\pi z}{b})~~m\in1,2,3,\ldots~~k_z=\frac{m\pi}{b}$\\
$X=\alpha_{n,m}e^{kx}+\beta_{n,m}e^{-kx}$\\
$k^2=k_y^2+k_z^2$\\
$\phi_{n,m}=(\alpha_{n,m}e^{kx}+\beta_{n,m}e^{-kx})\sin(\frac{n\pi y}{a})\sin(\frac{m\pi z}{b})$\\
Want to argue that any function satisfying $(1)$ and $(2)$ and $\lap\phi=0$ can be written as a superposition of solutions of this form.\\
(1.12.4) give Sturm-Lioville probems in the intervals $y\in(0,a)$, $z\in(0,b)$\\
That is these are eval problems for $k_y$, $k_z$\\
Solutions therefore give a basis of functions on $(0,a)$, $(0,b)$\\
This is just a fourier series representation.\\
Any function $f(y)$ such that $f(0)=f(a)=0$ can be written as $f(y)=\sum_n C_n \sin(\frac{n\pi y}{a})$\\
Hence any function $\phi(x,y,z)$ satisfying $(1)$ and $(2)$ can be written as $\phi=\sum_{m,n}X_{m,n}(x)\sin(\frac{n\pi y}{a})\sin(\frac{m\pi z}{b})$\\
Imposing $\lap\phi=0$ gives us $\phi=\sum_{m,n}\phi_{m,n}$ for some $\alpha_{m,n},~\beta_{m,n}$\\
Imposing $\phi\rightarrow 0$ as $x\rightarrow \infty$ implies that the $\alpha$'s are equal to 0 for all $m,n$\\
Imposing $\phi=\phi_0$ at $x=0$ requires $\phi_0=\sum^\infty_{m,n} \beta_{m,n}\sin(\frac{n\pi y}{a})\sin(\frac{m\pi z}{b}$\\
Use the orthogonality of these efunctions, with inner product
$$(\psi_y(y), \psi_z(y))=\int^\infty_0~dy \psi_y(y)\psi_z(y)$$
$$(\sin(\frac{p\pi y}{a}),\sin(\frac{q\pi y}{a})=\int^a_0~dy \sin(\frac{p\pi y}{a})\sin(\frac{q\pi y}{a})=\frac{a}{2}\delta_{pq}$$
So $\beta_{m,n}=\int^a_0~dy\int^b_0~dz \phi_0\sin(\frac{n\pi y}{a})\sin(\frac{m\pi z}{b})$, explaination below:
Take the inner product wrt. y, $(\phi_0,\sin(\frac{n^\prime\pi y}{a}))_y=\int^a_0~dy \phi_0\sin(\frac{n^\prime \pi y}{a})=\int^a_0~dy\sum_{m,n}\beta_{m,n}\sin(\frac{n\pi y}{a})\sin(\frac{m\pi z}{b})\sin(\frac{n^\prime\pi y}{a})=\frac{a}{2}\sum_m\beta_{m,n\prime}\sin(\frac{m\pi z}{b})$\\
Similarly with z, we get it equal to $\frac{ab}{4}\beta_{m,n}$\\
$\beta_{m,n}=\frac{4}{ab}\phi_0\int^a_0~dy \sin(\frac{n\pi y}{a})\int^b_0 \sin(\frac{m\pi z}{b})$\\
$=\frac{16}{mn\pi^2}\phi_0~~m,n=odd$\\
$=0$ otherwise\\
So 
$$\phi_0\sum_{m,n~odd}\frac{16}{mn\pi^2}\phi_0e^{-k_{m,n}x}\sin(\frac{n\pi y}{a})\sin(\frac{m\pi z}{b})~~~~~(1.12.5)$$
Notes:\\ 
- Linear Algebra is key: $\sin(\frac{n\pi y}{a})$ form an orthogonal basis.\\
- Adapt coordinate system to given boundries\\
- Given more general boundry condition, write $\phi$ as a superposition of solutions to simple boundry conditions. i.e. we can do them making all but 1 zero,  solve, then work out another one, setting the rest to zero. Then add them all.\\
\\
If we do this in spherical coordinates we can say that $\phi=\phi_0(\theta)$.\\
Find $\phi$ inside a spherical shell of radius $a$ with $\phi(r=a,\theta,\phi)=\phi_0(\theta)$\\
$\lap\phi=0$ for $r<a$\\
In spherical polars,
$$\dfrac{1}{r^2}\pd{}{r}\left(r^2\pd{\phi}{r}\right)+\dfrac{1}{r^2\sin\theta}\pd{}{\theta}\left(\sin\theta\pd{\phi}{\theta}\right)+\dfrac{1}{r^2\sin^2\theta}\pd{^2\phi}{\varphi^2}=0$$\\
By symmetry, assume $\pd{\phi}{\varphi}=0$\\
Look for a product solution\\
$\phi=R(r)\Theta(\theta)$\\
$\od{}{r}\left(r^2\od{R}{r}\right)=cR=l(l+1)R$\\
$\od{}{\theta}\left(\sin\theta\od{\Theta}{\theta}\right)=-c\sin(\theta) \Theta=-l(l+1)\sin\theta\Theta$\\
Solution: $R(r)=A_lr^l+\dfrac{B_l}{r^{l+1}}$\\
- doesn't fix $l$, because I can't impose the boundry condition at $r=a$ for individual product solutions.\\
In $\theta$, set $x=\cos\theta$\\
$\od{}{x}\left((1-x^2)\od{\Theta}{x}\right)+l(l+1)\Theta(x)=0$\\
Legendre's Eqn: Want to solve this for $x\in(-1,1)$ subject to the b.c. that $\Theta(x)$ is regular at $x=\pm1$\\
Generic solutions of this equation blows up at $x=\pm 1$\\
Therefore, eval problem: need tp find $l$ s.t. $\exists~\Theta_l(x)$ satisfying b.c.\\
We get that $l=0,1,2,3,\ldots$\\
Corresponding functions are called Legendre polynormials, $P_l(x)$\\
$P_o(x)=1$ (just a constant, by convention we normalize it)\\
$P_1(x)=x$\\
$P_2(x)=\frac{1}{2}(3x^2-1)$\\
etc\\
Legendre polynormials form a basis. Any regular function of $x\in(-1,1)$ can be written as
$$F(x)=\sum^\infty_{l=0}F_l P_l(x)$$
and they are orthogonal
$$\int^1 _{-1}P_l(x)P_{l^\prime}(x)~dx=\dfrac{2}{l+1}\delta_{ll^\prime}$$
Can then write a general solution as 
$$\phi=\sum_{l=0}^\infty\left(A_lr^l+\dfrac{B_l}{r^{l+1}}\right)P_l (\cos\theta)$$
b.c. in $r$:\\
$r=0,~~\phi$ is regular\\
$r=a,~~\phi(a,\theta)=\phi_0(\theta)$\\
The b.c. at $r=0$ implies that $B=0~\forall l$\\
The b.c. at $r=a$ implies that $\phi_0(\theta)=\sum^\infty_{l=0}A_la^lP_l(\cos\theta)$\\
So need to know Legendre polynormial expansion for $\phi_0(\theta)$\\
To determine $A_l$, integrate: $\int^1_{-1}~dxP_l(x)\phi_0(\theta)$\\
$\int^1_{-1}~dx P_l(x)\phi_0(\theta)=\sum_{l^\prime=0}^\infty A_{l^\prime}a^{l^\prime}\int^1_{-1}P_{l^\prime}P_l(x)$\\
$=\sum^\infty_{l^\prime=0}A_{l^\prime}a^{l^\prime}\dfrac{2}{l+1} \delta_{ll\prime}=\dfrac{2a^l}{l+1}A_l$\\
$$A_l=\dfrac{(l+1)}{2a^l}\int^1_{-1}~dx P_l(x)\phi_0(\theta(x))$$
$x=\cos\theta$, $dx=-\sin\theta~d\theta$\\
$$A_l=\dfrac{l+1}{2a^l}\int^\pi_0~d\theta\phi_o(\theta)P_l(\cos\theta)$$
If $\pd{\phi}{\varphi}\neq0$, need a basis in $\varphi\in(0,2\pi): e^{\pm im\varphi}$\\
Legendre polynormials are replaces by
$$P_l^m(x)=(-1)^m(1-x^2)^{\frac{m}{2}}\od{^m}{x^m}P_l(x)$$
This also means that $l\geq m$\\
Gives us a basis for functions on the sphere: Spherical Harmonics:
$$Y_{lm}(\theta,\varphi)=\sqrt{\dfrac{2l+1}{4\pi}\dfrac{(l-m)!}{(l+m)!}}P_l^m(\cos\theta)e^{im\varphi}$$
Basis: any function on the sphere can be written as $f(\theta,\varphi)=\sum^\infty_{l=0}\sum^l_{m=-l}c_{lm}Y_{lm}(\theta,\varphi)$\\
Orthonormal: 
$$\int^{2\pi}_0~d\varphi \int^\pi_0 \sin\theta~d\phi~Y^{*}_{l^\prime m^\prime}(\theta,\varphi)Y_{lm}(\theta,\varphi)=\delta_{ll^\prime}\delta_{mm^\prime}$$


\end{document}
