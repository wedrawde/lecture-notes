\begin{center}

Lecture 8

\end{center}

Define the quadropole moment: $$q_{ij} = \int \left(3 x_i' x_j' - \delta_{ij} |\vec{r}|^2\right) dV'$$

This is the multipole expansion for $\phi$. The expansion simplifies the calcuation of $\phi$: you just need to workout the multipole moments once.

Also note that $q_{ij} = q_{ji}$ and $\sum_i q_ii = 0$.

To go to higher orders it is better to use sphereical harmonics.

For a point dipole, the dipole moment is $\vec{p} = \int \vec{r}' \rho(\vec{r}') dV' = \frac{\vec{\epsilon}}{2} + q + \frac{-\vec{\epsilon}}{2} (-q) = q \vec{\epsilon}$