\begin{center}

Lecture 4

\end{center}

\subsection{Applications to symmetric sitations}

\begin{enumerate}
\item Sphereical symmetry. Suppose $\rho = \rho(r)$. The symmetry implies $\vec{E} = E(r) \vec{e}_r$ This is a function of only $r$. If $E(\theta)$ or $E(\phi)$ are not zero this would break sphereical symmetry by picking a direction.

Calculate $E(r)$ by Gauss' Law. Consider a surface $S$ which is a sphere of radius $r$.
\begin{align*}
\int_S \vec{E}\cdot \vec{n} ~dS &= E_r(r) \int \,\, dS = 4 \pi E_r(r) \\
&= 4 \pi Q_{encl} = 4\pi \int _V \rho (r) ~dV
\end{align*}

Thus: 
\begin{equation}
E_r(r) = \frac{Q_{encl} (r)}{r^2}
\end{equation}

An example: consider an electric field for a charged shell of radius $a$ totally charge $Q$. 
\[
E(r) = \begin{dcases*}
0 & r < a \\
\frac{Q}{r^2} & r > a
\end{dcases*}
\]

Another example: Uniformly charged sphere of radius $a$, total charge $Q$.

$\implies$ $E = \frac{Q}{r^2}$ for $r > a$.

For $r<a$, $Q_{encl} = \int \rho dV = \rho \int_0^r dV = \rho \int_0^r r^2 \sin \theta dr d\theta d\phi = \rho \frac{1}{3} r^3 4 \pi$.

So $Q_{encl} (r) = \frac{4 \pi r^3}{3} \rho$ and $Q_{encl} = \frac{4\pi a^3}{3} \rho$ and so $Q_{encl} = \frac{r^3}{a^3} Q$.

Thus: \[
E(r) = \begin{dcases*}
\frac{r}{a^3} & r<a \\
\frac{Q}{r^2} & r>a 
\end{dcases*}
\]

Note $\vec{E}$ is continuous at $r=a$ when there is no surface charge density.

\item Cylindrical symmetry.

In cylindrical polars $(r, \phi, z)$ assume $\rho(r)$. Again symmetry gives $\vec{E} = \vec{E}_r (r) \vec{e}_r$ \footnote{Assuming we also preserve the reflection symmetries $z \rightarrow -z$ and $\phi \rightarrow -\phi$}.

Take $S$ to be a cylinder of radius $r$: 
\begin{align*}
\oint_S \vec{E}\cdot {n} ~dS &= E_r(r) \int dS = 2\pi r E_r(r) \int dz \\
&= 4 \pi Q_{encl} = 4 \pi \int^r \rho(\bar(r)) dV \\
&= 4 \pi \int^r \rho(\bar(r)) \bar(r) d\bar(r) d\phi dz \\
&= 8 \pi^2 (\int^r \rho(\bar(r)) \bar(r) d\bar(r))(\int dz) 
\end{align*}

And thus
\begin{equation} 
E_r(r) = \frac{4\pi}{r} \int^r \rho (\bar(r) \bar(r) d\bar(r))
\end{equation}

For example: for a uniformly charged cylinder of charge/unit length $\lambda$ and radius $a$: $$\lambda = \int^a \rho r dt d\phi = 2 \pi \frac{1}{2} \rho a^2 = \pi a^2 \rho$$.

For $r>a$: $$E_r(r) = \frac{4 \pi}{r} \rho \int^a \bar{r} d\bar{r} = \frac{2 \pi \rho a^2}{r} = \frac{2 \lambda}{r}$$

And for $r<a$: $$E_r(r) = \frac{4 \pi}{r} \rho \int^a \bar{r} d\bar{r} = 2 \pi \rho r = \frac{2 \lambda r}{a^2}$$

\item Planar symmetry (something like a charged sheet)

$\rho(z)$ then by symmetry: $E = E(z) \vec{e}_z$ (but often allow components $E_x$, $E_y$, not determined by charge density). Draw a Guassian surface with faces $z=z_+$, $z=z_-$, $r^2 = (x^2 + y^2) = a^2$.

Now $\oint_S \vec{E}\cdot {n} ~dS = (E_z (z_+) - E_z(z_-)) \cdot \pi a^2 = 4 \pi Q_{encl} = 4 \pi \int \rho(z) dV = 4 \pi \int_{z_-}^{z_+} \rho (z) dz \cdot \pi a^2$

So $E_z (z_+) - E_z(z_-) = 4 \pi \int_{z_-}^{z_+} \rho(z) dz$

\end{enumerate}