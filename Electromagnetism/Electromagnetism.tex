\documentclass[a4paper,11pt]{article}

\usepackage[utf8]{inputenc}
\usepackage{amsfonts,amsmath,amsthm,amssymb}
\usepackage{graphicx}
\usepackage[english]{babel}
\usepackage[T1]{fontenc}
\usepackage[urw-garamond]{mathdesign}
\usepackage{hyperref}
\usepackage{fullpage}
\usepackage{indentfirst}
\usepackage{caption}
\usepackage{wrapfig}
\usepackage{url}
\usepackage{biblatex}
\usepackage{mathtools}
\renewcommand{\leq}{\leqslant}
\renewcommand{\geq}{\geqslant}

%Make compatible with copy pasteing from Corey's tex
\newcommand{\sech}{\mathrm{sech} \,}
\newcommand{\bra}[1]{\langle #1|}
\newcommand{\ket}[1]{|#1\rangle}
\newcommand{\braket}[2]{\langle #1|#2\rangle}
\newcommand{\set}[1]{\{#1\}}
\newcommand{\pd}[2]{\dfrac{\partial #1}{\partial #2}}
\newcommand{\od}[2]{\dfrac{d #1}{d #2}}
\newcommand{\Nab}{\vec{\nabla}}
\newcommand{\dop}[2]{\vec{#1}\cdot\vec{#2}}
\newcommand{\lap}{\nabla^2}
\newcommand{\dprime}{\prime\prime}
%See repo for his version

\title{Electromagnetism}

\date{}

\author{}

\begin{document}

\maketitle

\begin{center}

Lecture 1

\end{center}

\section*{Introduction}

\subsection*{What is a dynamical system?}

A dynamical system is a set of functions which depend on ``time'', a single variable $t$. Time can be discrete $t=0,1,2,3,\ldots$ and therefore our dynamical variable will look like $x(t)$ and thus $x(0)=x_0$, $x(1)=x_1$ and so forth. To specify how $x_i$ ($i \in \mathbb{Z}$) we need some `law'.

\subsubsection*{Example}

Variables $x_i$, $y_i$. Some initial conditions $x_0 = 1$ and $y_0 = 1$.

\vspace{\baselineskip}

``Law of evolution'' $y_n = x_{n-1}$ and $x_n = y_{n-1} + x_{n-1}$.

This is a discrete time dynamical system or a `recurence relation'. This dynamical system produces the fibonacci numbers.

\subsubsection*{Logistic Difference Equation}

This could be considered as a population model.  $ x_{n+1} = a x_n - b x_n^2$ with $a,b > 0$ and constants.

This can be rescaled to give: $$x_{n+1} = r x_n (1-x_n)$$

\vspace{\baselineskip}

Time can also be continuous. Variables like $x(t)$ and $y(t)$. A dynamical system will have initial conditions like $x(0) = 1$ and $y(0) = 2$. The law of evolution are now differential equations. For example: $\frac{dy}{dt} = - x^2$ and $\frac{dx}{dt} = y + xy^59$. 
\begin{center}

Lecture 2

\end{center}


\begin{center}

Lecture 3

\end{center}


\begin{center}

Lecture 4

\end{center}

Consider: $$a(x,y) u_x + b(x,y) u_y + c(x,y) u = d(x,y)$$.

The characteristic projections are: $\frac{dx}{dt} = a$, $\frac{dy}{dt} = b$ which gives $\frac{dy}{dx} = \frac{b}{a}$.

Along the characteristic projections: $\frac{du}{dt} = u_x \frac{dx}{dt} + u_y \frac{dy}{dt} = a u_x + b u_y = d - cu$. These are now ODEs.

\subsubsection*{Example}

Solve $u_x + u_y + u = 1$ with $u=x^2$ on $y=0$.

\vspace{\baselineskip}

The idea is to transform to new coordinated $(\nu, \xi)$ where $\nu$ is constant along the characteristic projections. The transformed equations will then be an ODE in $\xi$. The characteristic projections are then $\frac{dy}{dx} = 1$ $\implies$ $x-y = \text{constant}$ so take $\nu = x-y$.

Chose $\xi = x$ (independent of $\nu$). The chain rule gives $u_x = u_{\xi} \xi_x + u_{\nu} \nu_x = u_{\xi} + u_{\nu}$ and $u_y = u_{\xi} \xi_y + u_{\nu} \nu_y = - u_{\nu}$.

So the PDE is $u_{\xi} + u_{\nu} - u_{\nu} + u = 1$ which gives $u_{\xi} = 1 - u$. Integrate to get: $$\log (1-u) = -\xi + f(\nu)$$.

Thus $$u(\nu, \xi) = 1 - g(\nu) e^{\xi}$$

Transform back to $(x,y)$ coordinates to get the general solution: $$u(x,y) = 1 - g(x-y) e^{-x}$$

To find the particular solution, fix $g$ with the Cauchy data: $u(x,0) = 1 - g(x) e^{-x} = x^2$ $\implies$ $g(x) = (1-x^2)e^x$. Thus the particular solution is: $$u(x,y) = 1 - (1-(x-y)^2) e^{x-y} e^{-x} = 1 - (1-(x-y)^2) e^{-y}$$

\subsection{Acceleration Waves}

Another property of characteristic projections: curves across which $u_x$ or $u_y$ can be discontinuous. 

\subsubsection*{Example}

Solve: $u_x + cu_y = 0$ with $c$ constant and $u(x,0) = -x$ for $x<0$ and $u(x,0) = x$ for $x>0$.

\vspace{\baselineskip}

The characteristic projections satisfy $\frac{dx}{dt} = 1$, $\frac{dy}{dt} = c$ and along them $\frac{du}{dt} = 0$. So $u = \text{const}$ along the characteristic projections which are straight lines $y=cx + \text{const}$

The discontinuity in $u_x$ at $x=0$ propagates along the line $y=cx$. This is a ``weak solution'', i.e. one which satisfies the PDE everywhere except on some curve(s). This is as opposed to a classical solution where $u, u_x, u_y$ exist everywhere and satisfy the PDE.

Claim: Any curve $C = (x(\tau), y(\tau))$ across which $u_x$ or $u_y$ are discontinuous but $u$ is continuous must be a characteristic projection.

\begin{proof}

Differentiate along $C$: $\frac{du^{+}}{d\tau} = u_x^{+} \frac{dx}{d\tau} + u_y^{+} \frac{dy}{d\tau}$ and $\frac{du^{-}}{d\tau} = u_x^{-} \frac{dx}{d\tau} + u_y^{-} \frac{dy}{d\tau}$ where $u^+, u^-$ are the limiting values as $C$ is approached from either side. Since $u$ is continuous across $C$, thus $u^+ = u^-$ so $\frac{d}{d\tau} (u^+ - u^-) = 0$. Therefore: $$\underbrace{(u^-_x + u^+_x)}_{[u_x]} \frac{dx}{d\tau} + \underbrace{(u^-_y + u^+_y)}_{[u_y]} \frac{dy}{d\tau} = 0$$ with $[u_x]$ being the ``jump'' in $u_x$.

Both $u^+$ and $u^-$ are classical solutions of the PDE expect on $C$ so $a[u_x] + b[u_y] + \underbrace{c[u]}_{=0} = 0$. These two equations are a pair of homogeneous linear equations for $[u_x]$ and $[u_y]$. Hence $[u_x]$, $[u_y]$ can only be none zero if $$0 = \begin{pmatrix} \frac{dx}{d\tau} & \frac{dy}{d\tau} \\ a & b \end{pmatrix} = b \frac{dx}{d\tau} - a \frac{dy}{d\tau}$$

That is, along $C$, $\frac{dy}{dx} = \frac{b}{a}$ and thus $C$ is a characteristic projection.

\end{proof} 
\begin{center}

Lecture 5

\end{center}

\subsection{Overview of types of objects in QM}

\begin{enumerate}
\item Numbers and functions (scalars)
\item Vectors $\ket{v}$ and adjoint vector $\bra{v}$.
\item Linear operators (these `live' in Hilbert space)
\end{enumerate}

Physical examples of these are:
\begin{enumerate}
\item Probability amplitude in $\mathbb{C}$, inner product of two vectors $\braket{v|w}$, norm $|v| = \sqrt{\braket{v|v}} \in \mathbb{R}$
\item State of a quantum system
\item Observables (Hermitian) and transformation (Unitary)
\end{enumerate}

Representation in n-dimensional Hilbert space:
\begin{itemize}
\item Specify an orthonormal basis $\set{\ket{j}} = \set{\ket{1}, \ket{2}, \ldots,\ket{n}}$ and $\braket{i|j} = \delta_{ij}$
\item A vector $\ket{v} = \displaystyle \sum_{j=1}^{n} v_j \ket{j}$ which we represent by an n-tuple: $\begin{pmatrix} v_1 \\ v_2 \\ \vdots \\ v_n \end{pmatrix}$
\item The adjoint $\bra{v} = \displaystyle \sum_{j=1}^{n} v_j^{*} \bra{j}$ which we represent by an n-tuple: $\begin{pmatrix} v_1^{*} ,& v_2^{*} ,& \cdots ,& v_n^{*} \end{pmatrix}$
\end{itemize}

The bracket acts as: $$\braket{v|v} = \begin{pmatrix} v_1^{*}  ,& v_2^{*} ,& \cdots ,& v_n^{*} \end{pmatrix} \begin{pmatrix} v_1 \\ v_2 \\ \vdots \\ v_n \end{pmatrix} = \sum_{j=1}^{n} |v_j|^2$$

The allowed operations on vectors are:

\begin{itemize}
\item Addition: $\ket{v} + \ket{w} = \sum_j (v_j + w_j)\ket{j}$
\item Scalar multiplication: $\alpha \ket{v} = \sum_j (\alpha v_j) \ket{j}$
\end{itemize}

The formulas for $\bra{v} + \bra{w}$ and $\alpha \bra{w}$ are similar. $\bra{v} + \ket{w}$ is ill defined.

\begin{itemize}
\item Inner product: $\braket{v|w}$ is a scalar.
\item Outer product: $\ket{v}\bra{w}$ is an operator (matrix).
\end{itemize}

How do we get a coefficient $v_k$ in terms of $\ket{v}$ and $\set{\ket{j}}$? Use that: $$\braket{k|v} = \sum_j v_j \underbrace{\braket{k|j}}_{\delta_{ij}} = v_k$$ $$\implies \ket{v} = \sum_j \ket{j} \braket{j|v}$$

Useful properties of inner products:

\begin{itemize}
\item Schwarz identity: $|\braket{v|w}| \leq |v| \cdot |w|$
\item Triangle inequality: $|v+w| \leq |v| + |w|$
\end{itemize}

\subsection{Linear Operators}

An operator $\hat{O}$ is an instruction for transforming a given vector $\ket{v}$ into another vector $\ket{v'}$. That is $\hat{O} \ket{v} = \ket{v'}$ and $\bra{v} \hat{O} = \bra{v'}$

A linear operator satisfies
\begin{itemize}
\item $\hat{O} (a \ket{v} + b \ket{w}) = a \hat{O} \ket{v} + b \hat{O} \ket{w}$
\item $(a \ket{v} + b \ket{w}) \hat{O} = a \ket{v} \hat{O} + b \ket{w} \hat{O}$
\end{itemize}

Note: once we know $\hat{O} \ket{j} = \ket{j'}$ for basis vectors we automatically know: $$\hat{O} \ket{v} = \hat{O} \left(\sum_j v_j \ket{j}\right) = \sum_j v_j \hat{O} \ket{j} = \sum_j v_j \ket{j'}$$

Examples of linear operators:
\begin{itemize}
\item Identity operator: $\hat{I}$ defined by $\hat{I} \ket{v} = \ket{v}$ and $\bra{v} \hat{I} = \bra{v}$ $\forall v$. In n dimensional Hilbert space, this can be represented by the n by n identity matrix.
\item Rotation operator in $\mathbb{R}^3$. For example $\hat{R}$ is a rotation around $\hat{z}$ axis by 90 degrees. We see that $\hat{R} \ket{1} = \ket{2}$, $\hat{R} \ket{2} = - \ket{1}$ and $\hat{R} \ket{3} = \ket{3}$. From these we can derive that $\hat{R}$ is $$\hat{R} = \begin{pmatrix} 0 & -1 & 0 \\ 1 & 0 & 0 \\ 0 & 0 & 1 \end{pmatrix}$$
\end{itemize}
\begin{center}

Lecture 6

\end{center}

%line 254 in corey's notes

Example: Calculate $\phi$ for a linear charge dist. of length $2d$ and linear charge density $\rho _{L}$.
We shall take the origin as the centre of the line, and use cylindrical polars.\\
$$\phi = \int \dfrac{\rho(\vec{r}^\prime)}{|\vec{r}-\vec{r}^\prime|} dV = \int^d _{-d} dz^\prime \dfrac{\rho_{L}}{|\vec{r}-\vec{r}^\prime|}=\int ^d _{-d} dz^\prime \dfrac{\rho_{L}}{[(r-r^\prime)^2(z-z^\prime)^2}]^\frac{1}{2} =\int^d _{-d} dz^\prime \dfrac{\rho _{L}}{[r^2 +(z-z^\prime)^2]^\frac{1}{2}}$$
We then substitute $t=z^\prime - z$\\
$$\int^{d-z} _{-d-z} \dfrac{\rho_{L}}{(r^2 + t^2)^\frac{1}{2}} ~dt$$
now letting $t=r\sinh\theta$, $dt=r\cosh\theta ~d\theta$\\
$$\int d\theta \rho_{L} = \rho_{L} arcsinh\frac{t}{r}$$
Now we get:\\
$\phi=\rho_{L} arcsinh\left(\dfrac{z^\prime -z}{r}\right) \left|^d _{-d} \right.$\\
$=\rho_{L} \ln\left(\dfrac{z^\prime - z}{r}+\sqrt{\frac{z^\prime-z)^2}{r^2}+1} \right )\left|^{z^\prime =d} _{z^\prime = -d} \right.$\\
$$=\rho_L \ln(\dfrac{\sqrt{(z-d)^2 +r^2}+d-z}{\sqrt{(z+d)^2 +r^2}-(d+z)})$$
For $z>>d,r$ then $(z-d)^2 >>r^2 \Rightarrow \sqrt{(z-d)^2 -r^2} \approx (z-d)(1+\frac{r^2}{2(z-d)^2})$\\
So $\phi \approx \rho_{L} \ln \frac{z+d}{z-d} \approx \rho_L \ln(1+\frac{2d}{z-d}) \approx \rho \frac{2d}{z-d}$\\ 
\\

\subsection{Electric Field Lines and Equipotentials}

Theses act as aids to visualization.

\textbf{Electric Field Lines} are curves such that $\vec{E}$ is tangent to the cruve at each point. \textbf{Equipotential Surfaces} are the surfaces for $\phi = \text{constant}$. Electric field lines start and end on charges. Equipotentials are perpendicular to field lines, because $\Nab \phi \perp \phi = \text{constant}$.

\subsubsection*{Example}

$\vec{E} = \vec{E}_0$ gives field lines which are straight in the direction of $\vec{E}_0$. For $\vec{E} = \frac{q}{r^2} \vec{e}_r$ with field lines going radially out and equipotentials being spheres.

\subsection{Dipoles}

saw in previous example that far away, $\phi \approx \frac{q}{r}$ $\implies$ $\vec{E} \approx \frac{q}{r^2}\vec{e}_r$. What happens if the total charge vanishes? The simplest example of this is a dipole. Consider a 
\begin{center}

Lecture 7

\end{center}

Examples of different types of equilibrium point.

\subsubsection*{Example 1}

$\dot{x} = y$ and $\dot{y} = - x$ has an equilibrium point at $x = y = 0$.

\subsubsection*{Example 2}

$\dot{x} = x$ and $\dot{y} = y$

Look at $\dot{x} = x$ $\implies$ $x = Ae^t$ and thus $y = B e^t$. Thus $\frac{y}{x} = \frac{B}{A}$ which is unstable.

\subsubsection*{Example 3}

$\dot{x} = - x$ and $\dot{y} = -2y$. Thus $y = \frac{B}{A^2}x^2$ and therefore we have parabolas through the origin as phase flows.

\subsubsection*{Example 4}

$\dot{x} = x$ and $\dot{y} = -y$. We have that $xy = AB$ and thus we have a saddle point.

\vspace{\baselineskip}

We can see there are lots of different behaviours near fixed points. To decide whether they are stable or not introduce the following definitions.

\subsection{Definitions of Stability}

An equilibrium point $x'$ is called Lyapunov stable iff $\forall \epsilon > 0, \exists \delta > 0 \text{such that if} |y-x'|<\delta \implies |g^t y - x'| < \epsilon$ recall $g^t y$ is $\phi(t,y)$ the solution to a dynamical system such that $\phi(0,y) = y$.

As an example the simple harmonic oscillator is Lyapunov stable because I can pick $\delta = \epsilon > 0$ and thus $|y-\underbrace{x'}_{0}| = |y| = |g^t y - x'|$.

Lyapunov stability can be thought of as ``start close stay close''.

\vspace{\baselineskip}

An equilibrium point $x'$ is quasi-asymptotically stable iff $\exists \delta > 0$ such that if $|y-x'|<\delta$ then $|g^t y - x'| \rightarrow 0$ as $t \rightarrow \infty$

\vspace{\baselineskip}

An equilibrium point is asymptotically stable iff it is both Lyapunov and quasi-asymptotically stable.

\subsubsection*{Example}

$\dot{x} = -x$ and $\dot{y} = -y$ and thus the equilibrium point is $x=y=0$. The solutions are $x=Ae^{-t}$ and $y = Be^{-t}$. Thus $r^2 = x^2 + y^2 = A^2 e^{-2t} + B^2 e^{-2t} = (A^2 + B^2) e^{-2t}$ and thus $r = \sqrt{A^2 + B^2} e^{-t}$. Therefore r is strictly monotonically decreasing $\dot{r} < 0$. This is Lyapunov stable since $|g^t y| < |y|$. Thus $|g^t y -x'| < |y-x'|$ and we can take $\delta = \epsilon$ which implies Lyapunov stable.

Now as $t \rightarrow \infty$ we see $r \rightarrow 0$ so $|g^t y - 0| \rightarrow 0$ $\implies$ $q$ is asymptotically stable.

\subsection{Some other terms}

Periodic orbits: $\exists T$ such that $g^{t+T} y = g^t y$ for some $y$

\subsection{Limit cycles}

Example of a limit cycle.

In polar coordinates pick $\dot{r} = r(1-r^2)$ and $\dot{\theta} = 1$. All trajectories apart from the origin come close to $r=1$ as $t\rightarrow \infty$. Thus $r=1$ is called a limit cycle.

\subsection{Invariant sets}

Critical points, periodic orbits, limit cycles are all examples of invariant sets. A set $S \subset M$ ($M$ is the phase space) is called invariant iff $x_0 \in S$ then $g^t x_0 \in S$ for all $t$.

\end{document}