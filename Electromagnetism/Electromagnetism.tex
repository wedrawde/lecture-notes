\documentclass[a4paper,11pt]{article}

\usepackage[utf8]{inputenc}
\usepackage{amsfonts,amsmath,amsthm,amssymb}
\usepackage{graphicx}
\usepackage[english]{babel}
\usepackage[T1]{fontenc}
\usepackage[urw-garamond]{mathdesign}
\usepackage{hyperref}
\usepackage{fullpage}
\usepackage{indentfirst}
\usepackage{caption}
\usepackage{wrapfig}
\usepackage{url}
\usepackage{biblatex}
\usepackage{mathtools}
\renewcommand{\leq}{\leqslant}
\renewcommand{\geq}{\geqslant}

%Make compatible with copy pasteing from Corey's tex
\newcommand{\sech}{\mathrm{sech} \,}
\newcommand{\bra}[1]{\langle #1|}
\newcommand{\ket}[1]{|#1\rangle}
\newcommand{\braket}[2]{\langle #1|#2\rangle}
\newcommand{\set}[1]{\{#1\}}
\newcommand{\pd}[2]{\dfrac{\partial #1}{\partial #2}}
\newcommand{\od}[2]{\dfrac{d #1}{d #2}}
\newcommand{\Nab}{\vec{\nabla}}
\newcommand{\dop}[2]{\vec{#1}\cdot\vec{#2}}
\newcommand{\lap}{\nabla^2}
\newcommand{\dprime}{\prime\prime}
%See repo for his version

\title{Electromagnetism}

\date{}

\author{}

\begin{document}

\maketitle

\begin{center}

Lecture 1

\end{center}

Lectures will use Guassian units, as opposed to SI units. That is $4 \pi \epsilon_0 = 1$.

\section*{Introduction}

The key details of the course are contained in Maxwell's equations:

\begin{align*}
\vec{\nabla} \cdot \vec{E} &= 4 \pi \rho \\
\vec{\nabla} \wedge \vec{E} &= \frac{-1}{c} \frac{\partial \vec{B}}{\partial t} \\
\vec{\nabla} \cdot \vec{B} &= 0 \\
\vec{\nabla} \wedge \vec{B} &= \frac{1}{c} \frac{\partial \vec{E}}{\partial t} + \frac{4 \pi}{c} \vec{J}
\end{align*}

% I REALLY need to define some things in this module
% Todo: get stuff from Corey

These are the punchline: most of the course is building up to them. They are an important part of our understanding of nature: this course is classical, but QED describes almost everything down to $10^{-18} m$.

Idea of a field: sources $\rho + \vec{j}$ effect fields locally, influence is transmitted with some finite speed.

\subsection*{Electric charges and currents}

The most basic thing we know is that electric charge is conserved. Fundamentally, charge is carried by carried by point particles: electrons, photons, etc. These are small, so classically better to use a smooth charge density $\rho (\vec{x}, t)$. The Total charge $Q$ is then $Q = \int_V \rho \,\, dV$ with $dV = dx dy dz = r \sin \theta dr d\theta \phi$.

A current is a moving charge, we describe it using a current density $\vec{j} (\vec{x}, t)$. For a collection of charges moving uniformly with velocity $\vec{v}$, $\vec{j} = \rho \vec{v}$. This measures the total charge through a surface per unit time. If we consider an arbitrary surface $S$ then the charge per unit time across the surface element $dS$ is $\vec{j} \cdot \vec{n} \, \, dS$. Thus the total current through $S$ is $I = \int_S \vec{j} \cdot \vec{n}  \, \,dS$. 

Consider a volume $V$ enclosed by a closed surface $S$. Charge conversation tells us that $I = \frac{-dQ}{dt}$. Therefore: $$\int \vec{j} \cdot \vec{n} \, \, dS = \frac{d}{dt} \int \rho dV$$.

Assuming $V$ is fixed gives $\int \vec{j} \cdot \vec{n} \, \, dS = - \int \frac{\partial \rho}{\partial t} dV$. Now using the divergence theorem: $$\int \vec{\nabla} \cdot \vec{j} dV = - \int \frac{\partial \rho}{\partial t} dV$$.

This must be true for all volumes $V$ and therefore:

\begin{equation}
\vec{\nabla} \cdot \vec{j} = - \frac{\partial \rho}{\partial t}
\end{equation}

This is the continuity equation.

\vspace{\baselineskip}

So far, we have discussed the general dynamical case, now we will specialise to statics. Therefore $\rho(\vec{x})$, $\vec{j} (\vec{x})$, $\vec{E} (\vec{x})$, $\vec{B} (\vec{x})$. Thus the continutity equation becomes $\vec{\nabla} \cdot \vec{j} = 0$ and Maxwell's equations become (in the static case):

\begin{align*}
\vec{\nabla} \cdot \vec{E} &= 4 \pi \rho \\
\vec{\nabla} \wedge \vec{E} &= 0 \\
\vec{\nabla} \cdot \vec{B} &= 0 \\
\vec{\nabla} \wedge \vec{B} &= \frac{4 \pi}{c} \vec{j}
\end{align*}

The equations for the electric and magnetic fields separate in the static case and can be studied separately.
\begin{center}

Lecture 2

\end{center}

\subsection{Well posed problems and Simple PDEs}

An important question with PDEs is what additional data are needed to give a unique solution?

Hadamard called a problem (PDE + data) well posed if:

\begin{enumerate}

\item It has a solution
\item The solution is unique
\item The solution depends continuously on the initial data.

\end{enumerate}

Continuous dependence on the data means small changes in the initial (or boundary) data lead to only small changes in the solution.

\subsubsection*{Example}

For Laplace's equation, let $u(x,t)$ solve the problem

$$u_{tt} + u_{xx} = 0$$

With $t > 0$, $u(x, 0) = 0$ and $u_t (x, 0) = 0$.

Let $v(x, t)$ solve the problem:

$$v_{tt} + v_{xx} = 0$$

For $t > 0$ and $v(x, 0) = 0$ and $v_t (x, 0) = \epsilon \sin \frac{x}{\epsilon}$.

\vspace{\baselineskip}

The first problem has solution $u(x,t) = 0$. The second problem has solution $v(x,t) = \epsilon^2 \sin \frac{x}{\epsilon} \sinh \frac{t}{\epsilon}$.
 
\begin{center}

Lecture 3

\end{center}

For any general surface we have that $\vec{E} \cdot \vec{n} \, \, dS$ is negative when the ray from $q$ to outside of the surface passes back into the volume. Thus when all of these contributions are added up we get cancellation for all of them other than one and obtain exactly the same result. If we consider $q$ outside of the surface then $\int \vec{E} \cdot \vec{n} \, \, dS = 0$ as the ray passes through an even number of times, and we have complete cancellation.

Now using the principle of superposition we see that 

\begin{equation}
\int_S \vec{E} \cdot \vec{n} \, \, dS = 4 \pi \sum_{i \in V} q_i
\end{equation}

where the sum is over the charges in $S$.

For continuous charge distributions:

\begin{equation}
\int_S \vec{E} \cdot \vec{n} \, \, dS = 4 \pi \int_V \rho \, \, dV
\end{equation}

Using the divergence theorem: $$ \int_S \vec{E} \cdot \vec{n} \, \, dS = \int_V \vec{\nabla} \cdot \vec{E} \, \, dV$$ so
$$\int_V \vec{\nabla} \cdot \vec{E} \, \, dV = 4 \pi \int_V \rho \, \, dV$$

This is true for any volume $V$ and thus 

\begin{equation}
\vec{\nabla} \cdot \vec{E} = 4 \pi \rho
\end{equation}

Given $\rho$, now we have two ways to get $\vec{E}$:

\begin{itemize}
\item Integrate $\vec{E}(\vec{r}) = \int \rho(\vec{r}) \frac{\vec{r} - \vec{r}'}{|\vec{r} - \vec{r}'}$
\item Solve Maxwell's equation: $\vec{\nabla} \cdot \vec{E}(\vec{r}) = 4 \pi \rho (\vec{r})$
\end{itemize}

In symmetrical situations, the latter (in the form of Guass Law) is usually easier.

Substitute the integral into the differential form:

$$\vec{\nabla} \cdot \vec{E} = \int \rho (\vec{r'}) \, \, \vec{\nabla} \cdot \frac{\vec{r} - \vec{r}'}{|\vec{r} - \vec{r}|^3} \, \, dV'$$

To evaluate $\vec{\nabla} \cdot \frac{\vec{r} - \vec{r}'}{|\vec{r} - \vec{r}|^3}$, take $\vec{r}'$ to be the origin. Now consider $\vec{\nabla} \frac{\vec{r}}{|\vec{r}|^3}$. Recall (from exercise sheet) $\vec{\nabla} \cdot \vec{r} = 3$ in 3D.

Now $\vec{\nabla} \cdot \frac{1}{| \vec{r} |^3} = \frac{-3 \vec{r}}{|\vec{r}|^5}$ for $\vec{r} \neq \vec{0}$

Then $$\vec{\nabla} \cdot \frac{\vec{r}}{|\vec{r}|^3} = \frac{\vec{\nabla} \cdot \vec{r}}{|\vec{r}|^3} + \vec{r} \cdot \vec{\nabla} \frac{1}{|\vec{r}|^3} = \frac{3}{|\vec{r}|^3} - \frac{3 \vec{r} \cdot \vec{r}}{|\vec{r}|^5} = 0$$ for $\vec{r} \neq \vec{0}$

To understand what happens at $\vec{r} = \vec{0}$, regulate the divergence. Thus $\frac{1}{|\vec{r}|^3} \rightarrow \frac{1}{(r+a)^{3/2}}$. Now compute: $\vec{\nabla} \cdot \frac{1}{(r+a)^{3/2}}$. Consider the $x$ component: $\frac{\partial}{\partial x} \left(\frac{1}{(x^2 + y^2 + z^2 + a^2)^{3/2}}\right) = \frac{- 3/2}{(x^2 + y^2 + z^2 + a^2)^{5/2}} \cdot 2x = \frac{-2x}{(x^2 + y^2 + z^2 + a^2)^{5/2}}$

Now $\vec{\nabla} \cdot \left( \frac{\vec{r}}{(r^2 + a^2)^{3/2}}\right) = \frac{3}{(r^2 + a^2)^{3/2}} - \frac{3 \vec{r} \cdot \vec{r}}{(r^2 + a^2)^{5/2}} = \frac{3a^2}{(r^2 + a^2)^{3/2}}$ as $a \rightarrow 0$ then $\frac{3a^2}{(r^2 + a^2)^{3/2}} \rightarrow 0$ for $r neq 0$ and $\frac{3a^2}{(r^2 + a^2)^{3/2}} \rightarrow \infty$ for $r = 0$.

Then $\int \vec{\nabla} \cdot \left(\frac{\vec{r}}{(r^2 + a^2)^{5/2}}\left) \, \, dV = 4 \pi \int_0^{\infty} \frac{3a^2}{(r^2 + a^2)^{5/2}} r^2 \, \, dr$ Let $r = az$

Then $\int \vec{\nabla} \cdot \left(\frac{\vec{r}}{(r^2 + a^2)^{5/2}}\left) \, \, dV = 4 \pi \int_0^{\infty} \frac{3 z^2}{(z^2 + 1)^{5/2}} \, \, dz$ which is independent of $a$. Thus the integral is $4 \pi$.

So write $\lim_{a \rightarrow 0} \frac{3a^2}{(r^2 + a^2)^{5/2}} = 4 \pi \delta^3(\vec{r})$ where $\delta^3(\vec{r}) = 0$ for $\vec{r} \veq \vec{0}$ and $\int \delta^3 (\vec{r}) \, \, dV = 1$.

So $\vec{\nabla} \cdot \left(\frac{\vec{r}}{|\vec{r}|^3}\right) = 4 \pi \delta^3(\vec{r})$.

Shift by a constant: $\vec{\nabla} \cdot \frac{\vec{r} - \vec{r}'}{|\vec{r} - \vec{r}|^3} = 4 \pi \delta^3 (\vec{r} - \vec{r}')$.

The more mathematical way: believe the divergence theorem:

$$\int \vec{\nabla} \cdot \left(\frac{\vec{r}}{|\vec{r}|^3}\right) \, \, dV = \int_S \frac{\vec{r} \cdot \vec{n}}{|\vec{r}|^3} dS$$

Chose $S$ as a sphere of radius $r$. That is $\vec{r} \cdot \vec{n} = r$ and $dS = r^2 \sin \theta \, d\theta \, d\phi$ and therefore $\int \sin \theta d\theta d\phi = 4 \pi$.

Finally:

$$\vec{\nabla} \cdot \vec{E} = \int \rho(\vec{r}') \vec{\nabla} \frac{\vec{r} - \vec{r}'}{|\vec{r} - \vec{r}|^3} \, \, dV = 4 \pi \int \rho (\vec{r}) \delta^3 (\vec{r} - \vec{r}') dV = 4 \pi$$
\begin{center}

Lecture 4

\end{center}

If $X_i$ are extensive variables (other than $U$ and $S$). Define $Z_i = \frac{\partial S}{\partial X_i}$ and $Y_i = \frac{\partial U}{\partial X_i}$.

Maxwell relations are simply: $$\frac{\partial Y_i}{\partial X_j} = \frac{\partial Y_j}{\partial X_i}$$ and $$\frac{\partial Y_i}{\partial Y_j} = - \frac{\partial X_j}{\partial X_i}$$

\subsection{Specific heat, susceptibilities etc}

Specific heat: Measures change in entropy (equivalently by Claussius' relation the change in heat) with respect to temperature $C = T \frac{\partial S}{\partial T}$. Specific heat at a fixed volume: $T \left(\frac{\partial S}{\partial T}\right)_V$ and at a fixed pressure: $T \left(\frac{\partial S}{\partial T}\right)_P$.

We can also define specific heat directly from thermodynamic potentials: $$C_v = \left(\frac{\partial U}{\partial T}\right)_V$$ this follows from the first law $dU = TdS - PdV$ $$\implies \, \, \, \, \left(\frac{\partial U}{\partial T}\right)_V = T \left(\frac{\partial S}{\partial T}\right)_V$$

Susceptibility: If $(X,Y)$ are a pair of conjugate extensive-intensive variables: $$\xi = \frac{\partial X}{\partial Y}$$

Compressibility: $(P,V)$ with $$\kappa = - \frac{1}{V} \frac{\partial V}{\partial P}$$ usually we distinguish $$\kappa_T = - \frac{1}{V} \left(\frac{\partial V}{\partial P}\right)_T$$ as isothermal compressibility and $$\kappa_S = - \frac{1}{V} \left(\frac{\partial V}{\partial P}\right)_S$$ is adiabatic compressibility.

We call a constant $T$ isothermal and $S$ constant adiabatic (isentropic). $dS = 0 \implies \delta Q = 0$ by Claussius relation. 

\subsubsection*{Example: Q3 from problem sheet}

$S=c(UNV)^{1/3}$ derive the relations between $T, P, \mu$ and the extensive parameters.

\vspace{\baselineskip}

The first law states: $dU = TdS - PdV + \mu dN$

From these: $$T = \left(\frac{\partial U}{\partial S}\right)_{V,N}$$
$$P = - \left(\frac{\partial U}{\partial V}\right)_{S,N}$$
$$\mu = \left(\frac{\partial U}{\partial N}\right)_{S,V}$$

Now we can work out: $$U = \frac{S^3}{c^3 VN} = \alpha \frac{S^3}{VN}$$
$$T = 3 \alpha \frac{S^2}{VN} \implies T^3 = (3\alpha)^3 \frac{S^6}{(VN)^3}$$
$$P = \alpha \frac{S^3}{V^2 N} \implies P^2 = \frac{\alpha^2 S^6}{(V^2 N)^2}$$
$$\mu = - \alpha \frac{S^3}{VN^2}$$

Now using these we can get $$P = \sqrt{\frac{T^3N}{9 \alpha V}}$$ and $$\mu = - \sqrt{\frac{T^3 V}{3^3 \alpha N}}$$ and $$P = -\mu \frac{N}{V}$$

\subsubsection*{Example}

Show $$C_P - C_V = \frac{T V \alpha^2}{\kappa_T}$$

With $\kappa_T = - \frac{1}{V} \left(\frac{\partial V}{\partial P}\right)_T$ and $\alpha = \frac{1}{V} \frac{\partial V}{\partial T}$ and $C_P = T \left(\frac{\partial S}{\partial T}\right)_P$ and $C_V = T \left(\frac{\partial S}{\partial T}\right)_V$.

Start from the first law (with $\mu = 0$). $dU = TdS - PdV \implies TdS = dU + PdV$

Using $C_p = \left(\frac{\partial U}{\partial T}\right)_P$ and $C_V = \left(\frac{\partial U}{\partial T}\right)_V$ we see that $$TdS = C_V dT + T \left(\frac{\partial P}{\partial T}\right)_V dV$$ and $$TdS = C_P dT - T \left(\frac{\partial V}{\partial T}\right)_V dP$$.

Now using the chain rule for partials: $$\left(\frac{\partial P}{\partial T}\right)_V = - \frac{1}{\left(\frac{\partial T}{\partial V}\right)_P \left(\frac{\partial V}{\partial P}\right)_T} = \frac{\alpha}{\kappa_T}$$

So $TdS = C_V dT + \frac{\alpha T}{\kappa_T} dV$ and thus $TdS = C_P dT - \alpha TV dP$ now eliminate $dS$: $$(C_P - C_V) dT = \frac{\alpha T}{\kappa_T} dV + \alpha TV dP$$

If we treat $V$ and $P$ as independent variables: $$dT = \left(\frac{\partial T}{\partial V}\right)_P dV + \left(\frac{\partial T}{\partial P}\right)_V dP$$

So: $$\left[ (C_P - C_V) \left(\frac{\partial T}{\partial V}\right)_P - T \left(\frac{\partial P}{\partial T}\right)_V \right] + \left[(C_P - C_V) \left(\frac{\partial T}{\partial P}\right)_V - \left(\frac{\partial V}{\partial T}\right)_P \right] = 0$$

Now use the independence of $P,V$. This implies the coefficients of the exact differentials $dP, dV$ have to vanish: 
$$C_P - C_V = \frac{T \left(\frac{\partial P}{\partial T}\right)_V}{\left(\frac{\partial T}{\partial V}\right)_P} = \frac{T \alpha^2 V}{\kappa_T}$$
\begin{center}

Lecture 5

\end{center}

Now we can proove the Existance and Uniqueness Theorem. (That is: there exists a unique solution $\dot{x} = F(x)$ with initial condition $x(0) = x_0$ for $t \in [-a, a]$ for some $a > 0$ if $F$ is $C_1$)

%\begin{proof}

Assumptions:
\begin{enumerate}
\item F is $C_1$ on $O_{\rho}$ where $\rho$ is a closed ball center on $x_0$.
\item We have shown that $F$ is Lipschitz continous on $O_{\rho}$. That is $|F(y) - F(x)| < k |y-x|$.
\item Since $F$ is continuous on compact $O_{\rho}$ it has a maximum value there. That is $|F(x)| \leq M$
\item Given this we will take $a < \text{min} \left\{\frac{\rho}{M}, \frac{1}{k}\right\}$
\end{enumerate}

Prove ``constructively'', construct an infinite sequence of functions $u^i (t)$ for $i = 1,2,\ldots,\infty$. Such that $\lim_{i\rightarrow \infty} u^i(t) = u(t)$ where $x(t) = u(t)$ solves the differential equation.

We need one more Lemma: Suppose $u^k : \underbrace{J}_{\mathclap{For t \in [-a, a]}} \rightarrow \mathbb{R}^n$ for $k=0,1,2,\ldots,\infty$ is a sequence of continuous functions such that given any $\epsilon > 0$ there exists $N$ such that $p,q > N$ we have $\max{t \in J} | u^p(t) - u^q(t) | < \epsilon$ then there exists $u$ such that $\max_{t \in J} |u_k(t) - u(t)| \rightarrow 0$ as $k \rightarrow \infty$.

We shall construct a sequence of continous functions $u^k (t)$ for $k=0,1,2,\ldots,\infty$ using Picard iteration. To get a clue about how we do this turn our ODE into an integral equation.

\vspace{\baselineskip}

Consider the equation: $x(t) = x_0 + \int_0^t F(x(s)) \,\, ds$ Differentiate with respect to $t$: $\dot{x} = F(x(t))$ which is the equation we want to solve. The solution to the integral equation is the solution to the differential equation, but also: $x(0) = x_0 + \int_0^0 F(x(s)) \, \, ds = x_0$ so the integral equation also implies the initial condition $x(0) = x_0$.

Define the Picard iteration. Define a sequence of functions: $$u^{k+1}(t) = x_0 + \int_0^t F(u^k(s)) \, \, ds$$ define $u^0(t) = x_0$

To be able to use that $F$ is $C_1$ and Lipschitz continuous we need that $u^k(s) \in O_{\rho}$ $\forall k$ and all relevant $s$. $u^0(t) = x_0 \in O_{\rho}$ Now $$u^1(t) = x_0 + \int_0^t F(u^0(s)) \, \, ds$$.

Consider $$|u^1(t) - x_0| = |\int_0^t F(u^0(s)) \, \, ds| \leq \int_0^t |F(u^0(s))| \, \, ds \leq M \int_0^t ds \leq M|t|$$

We know that $|F(y)| < M$ $\forall y \in O_{\rho}$. Now $t\in[-a, a]$ so $|t| < a \leq \text{min} \left\{\frac{\rho}{M}, k\right\} \leq \frac{\rho}{M}$. Thus $$u^1 - x_0 \leq \frac{\rho}{M} M = \rho$$

Now $$|u^{k+1} - x_0| = |\int_0^t F(u^k(s)) \, \, ds| \leq \int_0^t |F(u_k(s))| ds \leq M \int_0^t ds = Mt \leq \rho$$

If $u^k(s) \in O_{\rho}$ for $|s| < |t| < a$ then $|F(u^k(s))| < M$ $\implies$ $u^{k+1} (t) \in O_{\rho}$ for $|t| < a$.
\begin{center}

Lecture 6

\end{center}

%line 254 in corey's notes

Example: Calculate $\phi$ for a linear charge dist. of length $2d$ and linear charge density $\rho _{L}$.
We shall take the origin as the centre of the line, and use cylindrical polars.\\
$$\phi = \int \dfrac{\rho(\vec{r}^\prime)}{|\vec{r}-\vec{r}^\prime|} dV = \int^d _{-d} dz^\prime \dfrac{\rho_{L}}{|\vec{r}-\vec{r}^\prime|}=\int ^d _{-d} dz^\prime \dfrac{\rho_{L}}{[(r-r^\prime)^2(z-z^\prime)^2}]^\frac{1}{2} =\int^d _{-d} dz^\prime \dfrac{\rho _{L}}{[r^2 +(z-z^\prime)^2]^\frac{1}{2}}$$
We then substitute $t=z^\prime - z$\\
$$\int^{d-z} _{-d-z} \dfrac{\rho_{L}}{(r^2 + t^2)^\frac{1}{2}} ~dt$$
now letting $t=r\sinh\theta$, $dt=r\cosh\theta ~d\theta$\\
$$\int d\theta \rho_{L} = \rho_{L} arcsinh\frac{t}{r}$$
Now we get:\\
$\phi=\rho_{L} arcsinh\left(\dfrac{z^\prime -z}{r}\right) \left|^d _{-d} \right.$\\
$=\rho_{L} \ln\left(\dfrac{z^\prime - z}{r}+\sqrt{\frac{z^\prime-z)^2}{r^2}+1} \right )\left|^{z^\prime =d} _{z^\prime = -d} \right.$\\
$$=\rho_L \ln(\dfrac{\sqrt{(z-d)^2 +r^2}+d-z}{\sqrt{(z+d)^2 +r^2}-(d+z)})$$
For $z>>d,r$ then $(z-d)^2 >>r^2 \Rightarrow \sqrt{(z-d)^2 -r^2} \approx (z-d)(1+\frac{r^2}{2(z-d)^2})$\\
So $\phi \approx \rho_{L} \ln \frac{z+d}{z-d} \approx \rho_L \ln(1+\frac{2d}{z-d}) \approx \rho \frac{2d}{z-d}$\\ 
\\

\subsection{Electric Field Lines and Equipotentials}

Theses act as aids to visualization.

\textbf{Electric Field Lines} are curves such that $\vec{E}$ is tangent to the cruve at each point. \textbf{Equipotential Surfaces} are the surfaces for $\phi = \text{constant}$. Electric field lines start and end on charges. Equipotentials are perpendicular to field lines, because $\Nab \phi \perp \phi = \text{constant}$.

\subsubsection*{Example}

$\vec{E} = \vec{E}_0$ gives field lines which are straight in the direction of $\vec{E}_0$. For $\vec{E} = \frac{q}{r^2} \vec{e}_r$ with field lines going radially out and equipotentials being spheres.

\subsection{Dipoles}

saw in previous example that far away, $\phi \approx \frac{q}{r}$ $\implies$ $\vec{E} \approx \frac{q}{r^2}\vec{e}_r$. What happens if the total charge vanishes? The simplest example of this is a dipole. Consider a 
\begin{center}

Lecture 7

\end{center}

Consider two charges $q$ and $-q$ a distance $\epsilon$ appart. Consider the distance from the center of the charges to be $\vec{r}$, the length from $q$ to be $r_{+}$ and the length from $-q$ to be $r_{-}$. Thus $r_{\pm}^2 = r^2 + \frac{1}{4} \epsilon^2 \mp \epsilon r \cos \theta$.

So
\begin{align*}
\frac{1}{r_{\pm}} &= \left(r^2 + \frac{1}{4} \epsilon^2 \mp \epsilon r \cos \theta\right)^{-\frac{1}{2}} \\
&= \frac{1}{r} \left(1 + \frac{1}{4}\frac{\epsilon^2}{r^2} \mp \frac{\epsilon}{r} \cos \theta\right)^{-\frac{1}{2}} \\
&\approx \frac{1}{r} \left(1 \pm \frac{\epsilon}{2r}\cos \theta\right) \,\, \, \, \text{for} \frac{\epsilon}{r} << 1
\end{align*}

So $$\phi \approx q \left(\frac{1}{r} \left(1 + \frac{\epsilon}{2r}\right) \right) - \frac{1}{r} \left(1 - \frac{\epsilon}{2r} \cos \theta\right)$$ then $$\phi \approx \frac{q\epsilon \cos \theta}{r^2} + O\left(\frac{\epsilon^2}{r^3}\right)$$ Now consider $\epsilon \cos \theta$ $$\epsilon \cos \theta = \frac{\vec{\epsilon} \cdot \vec{r}}{r^3}$$

So: $$\Phi \approx q \frac{\vec{\epsilon} \cdot \vec{r}}{r^3} + O\left(\frac{\epsilon^2}{r^3}\right)$$

Now let $\epsilon \rightarrow 0$ while holding $q$ fixed.

Define the dipole moment $\vec{p} = q \vec{\epsilon}$. Then 

\begin{equation}
\phi(\vec{r}) = \frac{\vec{p}\cdot \vec{r}}{r^3}
\end{equation}

This is the potential for the ideal dipole. Note $\phi ~ \frac{1}{r^2}$ as compared to $\frac{1}{r}$ for a charge. This is important for charge distributions with zero total charge, subleading correction for distribution with $Q_{\text{tot}} \neq 0$.

$$\vec{E} = - \Nab \phi = - \vec{r} \cdot \vec{r} \Nab \left(\frac{1}{|\vec{r}|^3}\right) - \frac{1}{|\vec{r}|^3} \Nab (\vec{p} \cdot \vec{r}) = \frac{3 \vec{p} \cdot \vec{r} \vec{r}}{|\vec{r}|^5} - \frac{\vec{p}}{|\vec{r}|^3}$$

If $\vec{p} = \begin{pmatrix} 0 & 0 & p \end{pmatrix}$ then $$E_x = \frac{3 p z x}{r^5}$$ $$E_y = \frac{3 p z y}{r^5}$$ $$E_z = \frac{3 p z^2}{r^5} - \frac{p}{r^3} = \frac{p (2x^2 - x^2 - y^2)}{r^5}$$

\subsection{Multipole expansion}

Monopole, dipole, quadropole, $\cdots$ We can build a succession of simple charge distributions with increasing rapid falloff for $\phi$. The idea is to use these simple forms for $\phi$ as an approximation to $\phi$ far away from some arbitrary localised charge distribution. That is $$\phi \approx \frac{q}{r} = \frac{2d \lambda}{r}$$ at $r >> d$. We want to see that $\phi = \frac{q}{r} + \frac{\vec{p}\cdot \vec{r}}{r^3} + \cdots$ for some $\vec{p}$

%Stealing from corey

Consider a charge distribution $\rho(\vec{r}^\prime)$ with support in a volume V. We want to approximate $\phi(\vec{r})$ where $|\vec{r}|>>|\vec{r}^\prime|$ for all $\vec{r}^\prime$ in V.
$$\phi(\vec{r})=\int \dfrac{\rho(\vec{r}^\prime)}{|\vec{r}-\vec{r}^\prime|}~dV,~\text{as we know}$$
We want to expand $\dfrac{1}{|\vec{r}-\vec{r}^\prime|}$ in powers of $\dfrac{|\vec{r}^\prime|}{|\vec{r}|}$\\
Recall the Taylor series expansion:\\
$f(\vec{r}-\vec{r}^\prime)\approx f(x,y,z)-x^\prime\partial_xf-y^\prime \partial_y f- z^\prime\partial_z f+...$\\
$\approx f(\vec{r}) -\vec{r}^\prime \cdot \vec{\nabla}f(\vec{r})+\frac{1}{2}\vec{r}^\prime \cdot \vec{\nabla}(\vec{r}^\prime \cdot \vec{\nabla} f(\vec{r}))$\\
So $\dfrac{1}{|\vec{r}-\vec{r}^\prime|} \approx -\vec{r}^\prime \cdot \vec{\nabla}(\frac{1}{|\vec{r}|}) + \frac{1}{2} \vec{r}^\prime \cdot \vec{\nabla}(\vec{r}^\prime\cdot \vec{\nabla}\frac{1}{|\vec{r}|})$\\
$\vec{\nabla}(\frac{1}{|\vec{r}|})=\frac{-\vec{r}}{|\vec{r}|^3}$, so second approx is:
$\frac{1}{|\vec{r}-\vec{r}^\prime|}=\vec{1}{|\vec{r}|}+\frac{\vec{r}\cdot\vec{r}}{|\vec{r}^\prime|^3}+\frac{1}{2}\vec{r}^\prime\cdot\vec{\nabla}(\frac{-\vec{r}^\prime \cdot \vec{r}}{|\vec{r}|^3})$ is \\
$\frac{1}{2} x^\prime _i \pd{}{x_i}(\dfrac{-x^\prime _j x_j}{|\vec{r}|^3})=\frac{1}{2} x^\prime _i (\dfrac{3x_i x_j ^\prime x_j}{|\vec{r}|^5} - \dfrac{x^\prime _j \delta_{ij}}{|\vec{r}|^3})$\\
$=\frac{1}{2}x^\prime _i x^\prime_j (\dfrac{3x_ix_j}{|\vec{r}|^5}-\dfrac{\delta_{ij}}{|\vec{r}|^3})+\ldots$\\
$\dfrac{1}{|\vec{r}-\vec{r}^\prime|}=\dfrac{1}{|\vec{r}|}+\dfrac{\vec{r}\cdot\vec{r}^\prime}{|\vec{r}|^3}+\frac{1}{2}(\dfrac{3(\vec{r}^\prime\cdot \vec{r})^2}{|\vec{r}|^5}-\dfrac{\vec{r}^\prime\cdot \vec{r}^\prime}{|\vec{r}|^3})$\\
So:\\
$\phi(\vec{r})=\frac{1}{|\vec{r}|}\int \rho(\vec{r^\prime})~dV +\dfrac{\vec{r}}{|\vec{r}|^3} \cdot\int \rho(\vec{r^\prime})\vec{r}^\prime ~dV + \dfrac{1}{2|\vec{r}|^5}\int \rho(\vec{r}^\prime)(3x_i ^\prime x_j ^\prime x_ix_j-x_i^\prime x_i ^\prime x_jx_j) ~dV$\\
$=\dfrac{1}{\vec{r}}\int \rho(\vec{r}^\prime)~dV+\dfrac{\vec{r}}{|\vec{r}|^3} \cdot\int \rho(\vec{r}^\prime) \vec{r}^\prime ~dV +\dfrac{x_i x_j}{2|\vec{r}|^5}\int \rho(\vec{r}^\prime)(3x_i ^\prime x_j ^\prime -\delta_{ij}x_k ^\prime x_k ^\prime)~dV$\\
\\
$$\phi(\vec{r})=\dfrac{q}{|\vec{r}|}+\dfrac{\vec{r}\cdot\vec{p}}{|\vec{r}|^3} + \dfrac{q_{ij}x_i x_j}{2|\vec{r}|^5}+\ldots$$\\

\end{document}