\begin{center}

Lecture 3

\end{center}

For light: $E=pc$ or $E = \hbar \omega$ and thus $p=\hbar k$. Recall $\hbar = \frac{h}{2\pi}$.

Using $E^2 = p^2 c^2 + m^2 c^4$ we see that photons are massless.

Even if we send photons through a double slit we still see an interference pattern as individual photons. The patern from the experiment with lots of photons acts as a probability density of where the photon will arrive.

Thus individual photons cannot move along well defined trajectories. Born suggested that each photon is associated with a wave, called probability amplitude $\Phi(x)$ such that $|\Phi(x)|^2$ measures the probability density of finding the photon at $x$. For a specified $p$, $\Phi(x)$ is a plane wave with wave number $k = \frac{p}{\hbar}$ and frequency $\omega = \frac{E}{\hbar}$

\vspace{\baselineskip}

Matter also exhibits wave like behaviour with a "de Brogile wavelength" given by $k=\frac{p}{\hbar}$. 

\subsection{Classical Limit}

The de Brogile wavelength is too tiny to see: for example, a pellet of 1g moving at 1cm/s $\implies$ $\lambda = \frac{2\pi}{k} = \frac{h}{p} \approx 10^{-26} cm$.

\subsection{Photoelectric Effect}

Electrons are liberated from a metal surface by shining light onto the surface. For sufficiently high frequency $\omega \geq \omega_0$ electrons are emitted at arbitrary high intensity and immediately. For low frequencies $\omega < \omega_0$ no electrons are emitted even at arbitrarily high intensity.