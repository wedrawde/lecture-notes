\begin{center}

Lecture 4

\end{center}

Superposition principle: If a quantum system $S$ can be in either of two states, then it can also be in any linear combination of them. This is called superposition.

Remarks: 

\begin{itemize}
\item This is not superposing classical waves, since it works for a single photon.
\item Photon path is de-localised $\rightarrow$ violates principle of perfect determination.
\item This is not due to our ignorance, it is part of the physics.
\end{itemize}

Complementarity principle: Complete knowledge of path is not compatible with presence of interference.

\vspace{\baselineskip}

Quantum states are neither classical waves nor classical particles, instead we characterise them as vectors in a vector space (which is a linear, complex, inner product space called ``Hilbert space''). We use Dirac notation $\underbrace{\ket{\Psi}}_{\text{state}}$