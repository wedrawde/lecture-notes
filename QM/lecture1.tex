\begin{center}

Lecture 1

\end{center}

QM :- 

\begin{itemize}

\item Origin in the 1920's to supplant classic mechanics. 
\item It has been confirmed theoretically and experimentally.
\item It has a wide range of applications.
	\begin{itemize}
	\item Chemistry (elections)
	\item Biological processes
	\item Information Sciences
	\item Quantum Computing
	\item Nano technology
	\item Cosmology
	\item Quantum Gravity
	\end{itemize}

\item Solid mathematical framework, but this has caused a conceptual revolution.

\end{itemize}

\section{Foundations}

\subsection{Classical Mechanics}

Classical Mechanics is founded on the following principles:

\begin{itemize}

\item Perfect determination (All observables have definite value at all time).

\item Measurement to arbitrary precision.

\item Continuity.

\item Separability: Independent systems do not affect each other.

Hence if we know the state of the system and the dynamics then we can predict precisely the state at any later time. 

$$ \frac{d}{dt} S = F[S(t)] $$

\end{itemize}

Classical mechanics concerns itself with two types of object: Particles and Waves.

\subsubsection{Particles}

The state of a single particle in 1-dimension is specified by position \& momentum. At an initial time $q_0 = q(t=0)$ and $p_0 = p(t=0)$. Dynamics then determines $q(t)$, $p(t)$ for all subsequent $t$. For $N$ particles in $d$ dimensions we have $n=Nd$ `degrees of freedom' for the system. These form the vectors $\{ \vec{q}_1, \vec{q}_2, \cdots, \vec{q}_N \}$ and $\{ \vec{p}_1, \vec{p}_2, \cdots, \vec{p}_N \}$. The state of the system is described by location in phase space $\Gamma$ described by the vectors of the dynamical variables. 

There are two convenient descriptions for time evolution of these systems:

\begin{enumerate}

\item Lagrangian: which produces $n$ 2$^\text{nd}$ order DEs

\item Hamiltonian: which produces $2n$ 1$^\text{st}$ order DEs

\end{enumerate}

The Lagrangian $L (q_i, \dot{q}_i)$ defines the action $\displaystyle S = \int_{t_0}^{t_1} L (q_i, \dot{q}_i) \, \, dt$. The principle of least action gives the Euler-Lagrange Equations:

$$\frac{d}{dt} \left ( \frac{\partial L}{\partial \dot{q}_i} \right ) = \frac{\partial L}{\partial q_i}$$